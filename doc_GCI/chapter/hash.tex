\chapter{Hash functions}
\label{hashfx}

\section{Configuration of hash algorithms}


\begin{center}

\begin{tabular}{| c | c|}
 \hline
 Algorithm 	&  Parameter \\
 \Gline
 MD5 	   	&  en\_gciHashAlgo\_Md5 \\
 \hline
 SHA1		&  en\_gciHashAlgo\_Sha1 \\
 \hline
 SHA224		&  en\_gciHashAlgo\_Sha224 \\
 \hline
 SHA256		&  en\_gciHashAlgo\_Sha256\\
 \hline
 SHA384		&  en\_gciHashAlgo\_Sha384 \\
 \hline
 SHA512		&  en\_gciHashAlgo\_Sha512 \\
\hline
 
\end{tabular}
\captionof{table}{Hash algorithms (en\_gciHashAlgo\_t)}
\label{tab:hash_config}

\end{center}

\section{Prototypes}

\subsection{Create a hash context}
\begin{lstlisting}
en_gciResult_t gciHashNewCtx( en_gciHashAlgo_t hashAlgo, GciCtxId_t* p_ctxID );
\end{lstlisting}

\begin{center}

\begin{tabular}{| c | *{3}{c|}}
 \hline
 Direction 	& Type 						& Parameter 					& Definition \\
 \Gline
 Input 	   	& en\_gciHashAlgo\_t	 	& hashAlgo				& Algorithm of the hash context
 \\
\hline
Output		& GciCtxId\_t* 				& p\_ctxID				& Pointer to the context's ID \\
\hline
 
\end{tabular}
\captionof{table}{Parameters for the creation of a hash
context}
\label{tab:hash_ctx}

\end{center}

\subsection{Update a hash context}
\begin{lstlisting}
en_gciResult_t gciHashUpdate( GciCtxId_t ctxID, const uint8_t* p_blockMsg, size_t blockLen );
\end{lstlisting}

\begin{center}

\begin{tabular}{| c | *{3}{c|}}
 \hline
 Direction 	& Type 						& Parameter 					& Definition \\
 \Gline
 Input 	   	& GciCtxId\_t			 	& ctxID					& Context's ID \\
\hline
Input		& uint8\_t* 				& p\_blockMsg			& Pointer to the block of the message \\
\hline
Input		& size\_t	 				& blockLen				& Block message's length \\
\hline
Return		& en\_gciResult\_t 			& en\_gciResult\_Ok		& When the data has been
added on success
\\
			& en\_gciResult\_t 			& en\_gciResult\_Err	& When error(s) occured \\
\hline
 
\end{tabular}
\captionof{table}{Parameters for the update of a new hash
context}
\label{tab:hash_upd}

\end{center}


\subsection{Clone a hash context}

The explication of the use of the clone of a hash context is done in
section \ref{ctx_clone}

\begin{lstlisting}
en_gciResult_t gciHashCtxClone( GciCtxId_t idSrc, GciCtxId_t* p_idDest );
\end{lstlisting}

\begin{center}

\begin{tabular}{| c | *{3}{c|}}
 \hline
 Direction 	& Type 						& Parameter 			& Definition \\
 \Gline
 Input 	   	& GciCtxId\_t			 	& idSrc					& The context which will be cloned \\
 \hline
 Output		& GciCtxId\_t*	 			& p\_idDest				& Pointer to the clone context ID \\
\hline
 
\end{tabular}
\captionof{table}{Parameters for the clone of a hash context}
\label{tab:hash_clone}

\end{center}

\subsection{Finish a hash context}
\begin{lstlisting}
en_gciResult_t gciHashFinish( GciCtxId_t ctxID, uint8_t* p_digest, size_t* p_digestLen );
\end{lstlisting}

\begin{center}

\begin{tabular}{| c | *{3}{c|}}
 \hline
 Direction 	& Type 						& Parameter 					& Definition \\
 \Gline
 Input 	   	& GciCtxId\_t			 	& ctxID					& Context's ID \\
 \hline
 Output	   	& uint8\_t*			 		& p\_digest				& Pointer to the digest of the complete added message \\
 \hline
Output		& size\_t*	 				& p\_digestLen			& Pointer to the length of the digest in bytes \\
\hline
 
\end{tabular}
\captionof{table}{Parameters for the calculation of the digest}
\label{tab:hash_fin}

\end{center}

\section{Steps to hash (Example)}

\begin{lstlisting}
#include <stdio.h>
#include <stdlib.h>
#include <string.h>
#include "crypto_iface.h"

int main(int argc , char *argv[])
{
    /* Error Management */
    en_gciResult_t err;

    /* MD5 context ID -> Always initialize to -1 */
    GciCtxId_t md5CtxID = -1, md5CloneCtxID = -1;

    /* Messages to hash */
    uint8_t a_data1[10] = {"Hello!"};
    uint8_t a_data2[30] = {"This is a Hash MD5 test"};
    uint8_t a_data3[10] = {"Thank you."};

    size_t data1Len = strlen(a_data1);
    size_t data2Len = strlen(a_data2);
    size_t data3Len = strlen(a_data3);

    int i;

    /* a MD5 digest is always 128 bits -> 16 bytes */
    uint8_t a_digest[GCI_MD5_SIZE_BYTES];

    /* Initialize the buffer */
    memset(a_digest, 0, GCI_MD5_SIZE_BYTES);

    size_t digestLen = 0;

    /* Create a new hash MD5 context */
    err = gciHashNewCtx(en_gciHashAlgo_MD5, &md5CtxID);

    /* Error coming from the creation of a new MD5-Hash context */
    if(err != en_gciResult_Ok)
    {
        printf("GCI Error in gciHashNewCtx: MD5");
    }

    /* Add the first data by updating the hash context */
    err = gciHashUpdate(md5CtxID, a_data1, data1Len);

    /* Error coming from the updating of the hash context with data1 */
    if(err != en_gciResult_Ok)
    {
        printf("GCI Error in gciHashUpdate: MD5");
    }

    /* Add the second data by updating the hash context */
    err = gciHashUpdate(md5CtxID, a_data2, data2Len);

    /* Error coming from the updating of the hash context with data2 */
    if(err != en_gciResult_Ok)
    {
        printf("GCI Error in gciHashUpdate: MD5");
    }

    /* Clone the context */
    err = gciHashCtxClone(md5CtxID, &md5CloneCtxID);
    if(err != en_gciResult_Ok)
    {
        printf("GCI Error in gciHashCtxClone: MD5");
    }

    /* Get the digest of this message */
    err = gciHashFinish(md5CtxID, a_digest, &digestLen);
    if(err != en_gciResult_Ok)
    {
        printf("GCI Error in gciHashFinish: MD5");
    }

    else
    {
        printf("GCI Info: Digest1 = ");
        for(i = 0; i < GCI_MD5_SIZE_BYTES; i++)
        {
            printf("%d", a_digest[i]);
        }
    }

    /* Initialize the buffer */
    memset(a_digest, 0, GCI_MD5_SIZE_BYTES);

    /* Add the third data by updating the hash context */
    err = gciHashUpdate(md5CloneCtxID, a_data3, data3Len);

    /* Error coming from the updating of the hash context with data3 */
    if(err != en_gciResult_Ok)
    {
        printf("GCI Error in gciHashUpdate: MD5");
    }

    /* Get the digest of this message */
    err = gciHashFinish(md5CloneCtxID, a_digest, &digestLen);
    if(err != en_gciResult_Ok)
    {
        printf("GCI Error in gciHashFinish: MD5");
    }

    else
    {
        printf("\r\nGCI Info: Digest2 = ");
        for(i=0; i<GCI_MD5_SIZE_BYTES; i++)
        {
            printf("%d, a_digest[i]);
        }

    }

    /* Delete the contexts */
    gciCtxRelease(md5CtxID);
    gciCtxRelease(md5CloneCtxID);

}
\end{lstlisting}