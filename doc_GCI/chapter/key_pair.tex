\chapter{Generation of key pair}

\section{Configuration of a key pair}

\begin{center}

\begin{tabular}{| c | c|}
 \hline
Algorithm				& Parameter \\
\Gline
RSA						& en\_gciKeyPairType\_RSA \\
\hline
DSA						& en\_gciKeyPairType\_DSA \\
\hline
ECDSA					& en\_gciKeyPairType\_ECDSA \\
\hline
\end{tabular}
\captionof{table}{Key pair types (en\_gciKeyPairType\_t) }
\label{tab:kp_type}

\end{center}



\begin{center}

\begin{tabular}{| c | c | c |}
\hline
Parameter		& Type \\				
\Gline
keyType			& en\_gciKeyPairType\_t \\
\hline
hash			& en\_gciHashAlgo\_t \\
\hline
keyPairParamRsa	& st\_gciRsaKeyGenConfig\_t*\\	
(If RSA is used as key pair type)	& \\
\hline
keyPairParamEcdsa	& st\_gciNamedCurve\_t \\
(If ECDSA is used as key pair type)	& \\
\hline	
keyPairParamDsa	& st\_gciDsaDomainParam\_t* \\
(If DSA is used as key pair type)	& \\
\hline

\end{tabular}
\captionof{table}{Configuration of key pair types
(st\_gciKeyPairConfig\_t))}
\label{tab:kp_conf}

\end{center}


\section{Prototype}

\begin{lstlisting}
en_gciResult_t gciKeyPairGen( const st_gciKeyPairConfig_t* p_keyConf,
GciKeyId_t* p_pubKeyID, GciKeyId_t* p_privKeyID );
\end{lstlisting}

\begin{center}

\begin{tabular}{| c | *{3}{c|}}
 \hline
 Direction 	& Type 							& Parameter 			& Definition \\
 \Gline
 Input 	   	& st\_gciKeyPairConfig\_t*	 	& p\_KeyConfig			& Pointer to the
 configuration of the key pair \\
 \hline
 Output	   	& GciKeyId\_t*			 		& p\_pubKeyID			& Pointer to the ID of the
 public key
 \\
 \hline
Output		& GciKeyId\_t* 					& p\_privKeyID			& Pointer to the ID of the private
key \\
\hline
\end{tabular}
\captionof{table}{Parameters for the generation of a key pair}
\label{tab:kp_gen}

\end{center}

\subsection*{RSA}

\begin{center}

\begin{tabular}{| c | *{2}{c|}}
 \hline
 Parameter 		& Configuration				& Exception(s) \\
 \Gline
 keyType   		& en\_gciKeyPairType\_Rsa 	& - \\
\hline
 modulusLen		& size\_t  					& less or equal than 0 \\					
 \hline

\end{tabular}
\captionof{table}{Configuration of RSA key pair
(st\_gciKeyPairConfig\_t)}
\label{tab:kp_rsa}

\end{center}

\subsection*{Digital Signature Algorithm (DSA)}

\begin{center}

\begin{tabular}{| c | *{2}{c|}}
 \hline
 Parameter 		& Configuration				& Exception(s) \\
 \Gline
 keyType   		& en\_gciKeyPairTypeDsa 	& - \\
\hline
 param			& st\_gciDsaDomainParam\_t	& less or equal than 0 \\					
 \hline
\end{tabular}
\captionof{table}{Configuration of DSA key pair
(st\_gciKeyPairConfig\_t)}
\label{tab:kp_dsa}

\end{center}

\subsection*{Elliptic Curve Digital Signature Algorithm (ECDSA)}

\begin{center}

\begin{tabular}{| c | *{2}{c|}}
 \hline
 Parameter 		& Configuration				& Exception(s) \\
 \Gline
 keyType   		& en\_gciKeyPairType\_Ecdsa & - \\
\hline
 param			& en\_gciNamedCurve\_t		& en\_gciNamedCurve\_Invalid \\					
 \hline
\end{tabular}
\captionof{table}{Configuration of ECDSA key pair
(st\_gciKeyPairConfig\_t)}
\label{tab:kp_ecdsa}

\end{center}

\section{Steps to generate a key pair}

\begin{lstlisting}
	#include <stdio.h>
	#include <stdlib.h>
	#include <string.h>
	#include "crypto_iface.h"

	int main(int argc , char *argv[])
	{

		/* Error management */
		en_gciResult_t err;
		
	    /* Configuration of a RSA key pair */
	    st_gciKeyPairConfig_t rsaConf = {.keyType = en_gciKeyPairType_Rsa
	    };
	    /* Length of the RSA modulus */
	    size_t rsaModLen = 1024;
	    rsaConf.un_keyPairParam.keyPairParamRsa->modulusLen = &rsaModLen;


	    /* ID of the RSA key pair */
	    GciKeyId_t rsaPubKeyID = -1;
	    GciKeyId_t rsaPrivKeyID = -1;

	    /* Generate the RSA key pair */
	    err = gciKeyPairGen(&rsaConf, &rsaPubKeyID, &rsaPrivKeyID);
	    
   		/* Error coming from the generation of a RSA key pair */
    	if(err != en_gciResult_Ok)
    	{
        	printf("GCI Error in gciKeyPairGen: RSA");
    	}
	}
\end{lstlisting}