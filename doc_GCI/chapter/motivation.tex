\chapter{Motivation}

Through the bachelor thesis ``Extension and Integration of an Abstract
Interface to Cryptography Providers'' done in the Institute of reliable Embedded
Systems and Communication Electronics (ivEsk), this Generic Cryptographic
Interface (GCI) has been designed to have a base of the main cryptographic
algorithms. The interface has to be as generic as possible to use different kinds of cryptographic
providers, which can be open-source cryptographic software
libraries or hardware-based cryptographic modules, and to
be used in different applications.
The institute wants to use hardware-based cryptographic module with one of its particularities of
key management. The interface shall have a key management too, so it can be possible to use it from
the hardware-based cryptographic module. When a cryptographic algorithm of the interface is used,
it has to be possible to configure it. The behavior of the algorithm should only be assigned by the
parameters coming from the configuration. No hidden states shall be used in the
interface's function, meaning that all parameters written in input of the function, as configuration, will be used for the
cryptographic algorithm and nothing else.
