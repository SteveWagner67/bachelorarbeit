\chapter{Interface}

\section{Initialization}

The initialization allows to initialized some part of the project if needed.
A user name and a password can be added if the initialization has to stay
private.

\begin{lstlisting}
en_gciResult_t gciInit( const uint8_t* p_user, size_t userLen, const uint8_t*
p_password, size_t passLen )
\end{lstlisting}

\begin{center}

\begin{tabular}{| c | *{3}{c|}}
 \hline
 Direction 	& Type 			& Parameter 		& Definition \\
 \Gline
 Input 	   	& uint8\_t*	 	& p\_user			& Pointer to the user name \\
\hline
Input 		& size\_t 		& userLen			& Length of the user name \\
\hline
Input		& uint8\_t		& p\_password		& Pointer to the password \\
\hline
Input 		& size\_t		& passLen			& Length of the password \\
\hline
 
\end{tabular}
\captionof{table}{Parameters for the initialization of the interface}
\label{tab:int_init}

\end{center}


\section{Delete the initialization of the interface}

When the project is going to the end, some parts have to be delete, i.e free the
memory used. This is the use of this part
	
\begin{lstlisting}	
en_gciResult_t gciDeinit( void )
\end{lstlisting}
	
	
