\chapter{Context management}

%%%%%%%%%%%%%%%%%%%%%%%%%%%%%%%%%%%%%%%%%%%%%%%%%%%%%%%%%%%%
%					DEFINITION							   %
%%%%%%%%%%%%%%%%%%%%%%%%%%%%%%%%%%%%%%%%%%%%%%%%%%%%%%%%%%%%
\section{Definition}

Several different configurations are possible for a cryptographic algorithm. These configurations
should not have parameters, which are implicitly written in the functions of the interface, but all
of them has to be configurable from the application part. Furthermore, data can be added to the
algorithm at any time, meaning that the state of the algorithm, the configuration and the previously
added data, have to be saved somewhere. The result can be computed, only when the application
needs it, that's why should all the data and the configuration be saved
somewhere too.
That's why the principle of the context is used. It represents the state of the
stateful algorithms.
Through the contexts, these informations, the configuration and the data, are encapsulated and
referred by an ID, which is used by the application when other data has to be added or when the
result has to be computed.
The contexts also allow to the application part to add data to an algorithm by only passing the ID
and this data at any time. The interface knows that this information has to be added to the context
with the referred ID. The context keeps the configuration and the data till it's
removed or a result is computed. When a result is computed, the context cannot be used again, because the data are
not saved in the interface, but added to a function from the provider and this one removes the data
when the result are computed. The context should then be removed and created again. This can be a
problem, but is solved and explained in the section \ref{ctx_clone}.
The cryptographic algorithms, which the principle of context is used, are:
\begin{itemize}[noitemsep]
  \item Hash, see section \ref{hashfx}
  \item Signature, see section \ref{sign}
  \item Message Authentication Code (MAC) \ref{sign}
  \item Symmetric cipher, see section \ref{cipher}
  \item Asymmetric cipher, see section \ref{cipher}
  \item Diffie-Hellman, see section \ref{dhKeys}
\end{itemize}


\section{Creation of a context}
Sometimes some parameters of a context need to change after the creation of a
context. That's why the ID of the context has to be initialized to -1. In this
case a context will be created, else the parameters who change since the
previous configuration will be saved in the context refered by the ID given
in input.

%%%%%%%%%%%%%%%%%%%%%%%%%%%%%%%%%%%%%%%%%%%%%%%%%%%%%%%%%%%%
%					CLONE CONTEXT						   %
%%%%%%%%%%%%%%%%%%%%%%%%%%%%%%%%%%%%%%%%%%%%%%%%%%%%%%%%%%%%

\section{Clone an existing context}
\label{ctx_clone}

One of the inconvenient of the interface comes from the finish part, where the
last calculation is done.
For the hash algorithm and the signature algorithm, when the digest/signature is
calculated, no more updates could be done with the last configuration and
with the previous updates.

The solution of this problem is the clone of the context.

When the digest/signature has to be calculated but the configuration and the
previous updates should be kept, the clone of the hash/signature context allows
to copy the configuration and the previous updates. Two contexts are identical
but one is use for the calculation of the digest/signature and the other one for
futur updates.

The principle of the clone is used for:
\begin{itemize}[noitemsep]
  \item Hash algorithm, see section \ref{hashfx}
  \item Signture/MAC algorithm, see section \ref{sign}
\end{itemize}

%%%%%%%%%%%%%%%%%%%%%%%%%%%%%%%%%%%%%%%%%%%%%%%%%%%%%%%%%%%%
%					DELETE CONTEXT						   %
%%%%%%%%%%%%%%%%%%%%%%%%%%%%%%%%%%%%%%%%%%%%%%%%%%%%%%%%%%%%
\section{Delete an existing context}
\label{delCtx}

When the context is not needed anymore, it can be removed and be used for an
other configuration, which can be completely different as the previous one.

Prototype:
\begin{lstlisting}
en_gciResult_t gciCtxRelease( GciCtxId_t ctxID );
\end{lstlisting}

\begin{center}

\begin{tabular}{| c | *{3}{c|}}
 \hline
 Direction 	& Type 						& Parameter				& Definition \\
 \Gline
 Input 	   	& GciCtxId\_t	 			& ctxID					& Context's ID \\
\hline
 
\end{tabular}
\captionof{table}{Parameters to delete a context}
\label{tab:ctx_rl}

\end{center}

