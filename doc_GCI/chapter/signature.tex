\chapter{Signature/MAC algorithms}
\label{sign}

\section{Configuration of Signature / MAC algorithms}

\begin{center}

\begin{tabular}{| c | c|}
 \hline
Algorithm				& Parameter \\
\Gline
RSA						& en\_gciSignAlgo\_RSA \\
\hline
DSA						& en\_gciSignAlgo\_DSA \\
\hline
ECDSA					& en\_gciSignAlgo\_ECDSA \\
\hline
MAC ISO9797 Algorithm 1	& en\_gciSignAlgo\_MAC\_ISO9797\_ALG1 \\
\hline
MAC ISO9797 Algorithm 3 & en\_gciSignAlgo\_MAC\_ISO9797\_ALG3 \\
\hline
Cipher-based MAC		& en\_gciSignAlgo\_CMAC \\
\hline
Hash-based MAC			& en\_gciSignAlgo\_HMAC \\
\hline
\end{tabular}
\captionof{table}{Signature/MAC algorithms (en\_gciSignAlgo\_t) }
\label{tab:sign_algo}

\end{center}



\begin{center}

\begin{tabular}{| c | c | c |}
\hline
Parameter		& Type \\				
\Gline
algo			& en\_gciSignAlgo\_t \\
\hline
hash			& en\_gciHashAlgo\_t \\
\hline
signConfigRsa	& st\_gciSignRsaConfig\_t \\	
(If RSA is used as algorithm)	& \\
\hline
signConfigCmac	& st\_gciSignCmacConfig\_t \\
(If CMAC is used as algorithm)	& \\
\hline	


\end{tabular}
\captionof{table}{Configuration of Signature/MAC algorithms
(st\_gciSignConfig\_t))}
\label{tab:sign_conf}

\end{center}



\section{Prototypes}
Two differents use of the signature are available.\newline
The first one is the generation of a signature, which will sign the datas
updated (see
section \ref{signGen} for more details.

The second one is the verification of a signature, which will sign the datas
updated but will at the end compared with this entered in the function (see
section \ref{signVfy} for more details).

\subsection{Creation of a context}

There is two possibilities of use for the signature context. The first one is to
generate a signature and the second one to verify a signature. It has been split
because some known provider needs to do the difference between the twice
possibilities.

\begin{lstlisting}
en_gciResult_t gciSignGenNewCtx( const st_gciSignConfig_t* p_signConfig,
GciKeyId_t keyID, GciCtxId_t* p_ctxID )
\end{lstlisting}

\begin{lstlisting}
en_gciResult_t gciSignVerifyNewCtx( const st_gciSignConfig_t* p_signConfig,
GciKeyId_t keyID, GciCtxId_t* p_ctxID )
\end{lstlisting}


\begin{center}

\begin{tabular}{| c | *{3}{c|}}
 \hline
 Direction 	& Type 						& Parameter 			& Definition \\
 \Gline
 Input 	   	& st\_gciSignConfig\_t*	 	& p\_signConfig			& Pointer to the
 configuration of the signature
 \\
 \hline
 Input	   	& GciKeyId\_t			 	& keyID					& ID of the key uses to sign \\
 \hline
Output		& GciCtxId\_t* 				& p\_ctxID				& Pointer to the
context's ID \\
\hline
\end{tabular}
\captionof{table}{Parameters for the creation of a signature/MAC context}
\label{tab:sign_ctx}

\end{center}

\subsubsection*{RSA}

The hash algorithm can be used if the
updated data has to be hashed before to be signed. If not, this parameter should be
configure as en\_gciHashAlgo\_None.

\begin{center}

\begin{tabular}{| c | *{2}{c|}}
 \hline
 Parameter 		& Configuration			& Prohibition(s) \\
 \Gline
 algo 	   		& en\_gciSignAlgo\_Rsa 	& - \\
\hline
 hash			& en\_gciHashAlgo\_t  	& en\_gciHash\_Invalid \\					
 \hline
 padding		& en\_gciPadding\_t 	& en\_gciPadding\_Invalid \\
 				&						& en\_gciPadding\_None \\
 \hline
\end{tabular}
\captionof{table}{Configuration of RSA Signature Scheme Algorithms
(st\_gciSignConfig\_t)}
\label{tab:sign_rsa}

\end{center}

\subsubsection*{DSA}

\begin{center}

\begin{tabular}{| c | *{2}{c|}}
 \hline
 Parameter 		& Configuration			& Prohibition(s) \\
 \Gline
 algo 	   		& en\_gciSignAlgo\_Dsa 	& - \\
\hline
 hash			& en\_gciHashAlgo\_t  	& en\_gciHash\_Invalid \\					
 \hline
\end{tabular}
\captionof{table}{Configuration of Digital Signature Algorithms
(st\_gciSignConfig\_t)}
\label{tab:sign_dsa}

\end{center}

\subsubsection*{ECDSA}

\begin{center}

\begin{tabular}{| c | *{2}{c|}}
 \hline
 Parameter 		& Configuration				& Prohibition(s) \\
 \Gline
 algo 	   		& en\_gciSignAlgo\_Ecdsa 	& - \\
\hline
 hash			& en\_gciHashAlgo\_t  		& en\_gciHash\_Invalid \\					
 \hline
\end{tabular}
\captionof{table}{Configuration of Elliptic Curve Digital Signature Algorithms
(st\_gciSignConfig\_t)}
\label{tab:sign_ecdsa}

\end{center}

\subsubsection*{Cipher-based MAC (CMAC)}

\begin{center}

\begin{tabular}{| c | *{2}{c|}}
 \hline
 Parameter 		& Configuration				& Prohibition(s) \\
 \Gline
 algo 	   		& en\_gciSignAlgo\_Cmac 	& - \\
\hline
 hash			& en\_gciHashAlgo\_t  		& en\_gciHash\_Invalid \\					
 \hline
 block			& en\_gciBlockMode\_t		& en\_gciBlockMode\_Invalid \\
 				&							& en\_gciBlockMode\_None \\
 \hline
 padding		& en\_gciPadding\_t			& en\_gciPadding\_Invalid \\
 \hline
 iv.data		& uint8\_t*					& NULL \\ 
 iv.len			& size\_t					& value less or equal than 0 \\	
 \hline
\end{tabular}
\captionof{table}{Configuration of Cipher-based MAC
(st\_gciSignConfig\_t)}
\label{tab:sign_cmac}

\end{center}

\subsubsection*{Hash-based MAC (HMAC)}

\begin{center}

\begin{tabular}{| c | *{2}{c|}}
 \hline
 Parameter 		& Configuration				& Prohibition(s) \\
 \Gline
 algo 	   		& en\_gciSignAlgo\_Hmac 	& - \\
\hline
 hash			& en\_gciHashAlgo\_t  		& en\_gciHash\_Invalid \\	
 				&							& en\_gciHash\_None \\				
 \hline
\end{tabular}
\captionof{table}{Configuration of Hash-based MAC
(st\_gciSignConfig\_t)}
\label{tab:sign_hmac}

\end{center}

\subsection{Update of the context}

The update of the context for generating and verificate a signature is the same.
The data have to be added to get a signature.

\begin{lstlisting}
en_gciResult_t gciSignUpdate( GciCtxId_t ctxID, const uint8_t* p_blockMsg,
size_t blockLen )
\end{lstlisting}

\begin{center}

\begin{tabular}{| c | *{3}{c|}}
 \hline
 Direction 	& Type 				& Parameter 			& Definition \\
 \Gline
 Input 	   	& GciCtxId\_t		& ctxID					& Context's ID \\
 \hline
 Input	   	& uint8\_t*			& p\_blockMsg			& Pointer to the message to sign \\
 \hline
 Input		& size\_t			& blockLen				& Length of message \\
 \hline
\end{tabular}
\captionof{table}{Parameters for the update of a signature/MAC context}
\label{tab:sign_upd}

\end{center}

\subsection{Clone of signature/MAC algorithm}

\begin{lstlisting}
en_gciResult_t gciSignCtxClone( GciCtxId_t idSrc, GciCtxId_t* p_idDest )
\end{lstlisting}


\begin{center}

\begin{tabular}{| c | *{3}{c|}}
 \hline
 Direction 	& Type 						& Parameter 			& Definition \\
 \Gline
 Input 	   	& GciCtxId\_t			 	& idSrc					& The context which will be cloned \\
 \hline
Output		& GciCtxId\_t*	 			& p\_idDest				& Pointer to the clone context ID \\
\hline
 
\end{tabular}
\captionof{table}{Parameters for the clone of a signature/MAC context}
\label{tab:sign_clone}

\end{center}

\subsection{Calculation / Verification of the signature}
After the data have been added to the context, if the context is to generate a
signature, than the signature will be compted and returned.
If the context if to verify a signature, than the signature to verify has to be
added to the function. Internally the signature with the updated data will be
computed but ot returned. Only if the the returning value of the function
indicates if the signatures are the same or not.

\begin{lstlisting}
en_gciResult_t gciSignGenFinish( GciCtxId_t ctxID, uint8_t* p_sign, size_t*
p_signLen )
\end{lstlisting}

\begin{center}

\begin{tabular}{| c | *{3}{c|}}
 \hline
 Direction 	& Type 				& Parameter 			& Definition \\
 \Gline
 Input 	   	& GciCtxId\_t		& ctxID					& Context's ID \\
 \hline
 Input	   	& uint8\_t*			& p\_sign				& Pointer to the generated signature \\
 \hline
 Input		& size\_t*			& signLen				& Pointer to the length of the generated
 signature \\
 \hline
\end{tabular}
\captionof{table}{Parameters for the computation of a signature/MAC}
\label{tab:sign_gen_fin}

\end{center}

\begin{lstlisting}
en_gciResult_t gciSignVerifyFinish( GciCtxId_t ctxID, const uint8_t* p_sign,
size_t signLen )
\end{lstlisting}

\begin{center}

\begin{tabular}{| c | *{3}{c|}}
 \hline
 Direction 	& Type 				& Parameter 			& Definition \\
 \Gline
 Input 	   	& GciCtxId\_t		& ctxID					& Context's ID \\
 \hline
 Input	   	& uint8\_t*			& p\_sign				& Pointer to the signature to verify \\
 \hline
 Input		& size\_t			& signLen				& Length of the signature to verify\\
 \hline
\end{tabular}
\captionof{table}{Parameters for the verification of a signature/MAC}
\label{tab:sign_vfy_fin}

\end{center}

\section{Step to generate a signature}
For this example the keys have already been added previously to the interface 
and an ID returned.
\begin{lstlisting}

	#include <stdio.h>
	#include <stdlib.h>
	#include <string.h>
	#include "crypto_iface.h"

	int main(int argc , char *argv[])
	{

	    /* Configuration of a RSA signature */
	    st_gciSignConfig_t signConfig = {.algo = en_gciSignAlgo_Rsa,
	        .hash = en_gciHashAlgo_None
	    };

	    signConfig.un_signConfig.signConfigRsa.padding = en_gciPadding_Pkcs1_Emsa;


	    /* Error Management */
	    en_gciResult_t err;

	    /* Messages to hash */
	    uint8_t a_data1[10] = {"Hello!"};
	    uint8_t a_data2[30] = {"This is a RSA Signature test"};
	    uint8_t a_data3[10] = {"Thank you."};

	    size_t data1Len = strlen(a_data1);
	    size_t data2Len = strlen(a_data2);
	    size_t data3Len = strlen(a_data3);

	    /* Buffer for the signature */
	    uint8_t a_signature[30];
	    size_t signLen;


	    int i;

	    /* RSA context ID */
	    GciCtxId_t rsaCtxID;

	    /* Init of the signature */
	    memset(a_signature, 0, 30);
	    signLen = 0;

	    /* Creation of the signature context with the RSA private key */
	    err = gciSignGenNewCtx(&signConfig, rsaPrivKeyID, &rsaCtxID);

	    /* Error coming from the creation of a new MD5-Hash context */
	    if(err != en_gciResult_Ok)
	    {
	        printf("GCI Error in gciSignGenNewCtx: RSA");
	    }

	    /* First update of the signature */
	    err = gciSignUpdate(rsaCtxID, a_data1, data1Len);

	    /* Error coming from the update of a RSA-signature context */
	    if(err != en_gciResult_Ok)
	    {
	        printf("GCI Error in gciSignUpdate: RSA");
	    }

	    /* Second update of the signature */
	    err = gciSignUpdate(rsaCtxID, a_data2, data2Len);

	    /* Error coming from the update of a RSA-signature context */
	    if(err != en_gciResult_Ok)
	    {
	        printf("GCI Error in gciSignUpdate: RSA");
	    }

	    /* Third update of the signature */
	    err = gciSignUpdate(rsaCtxID, a_data3, data3Len);

	    /* Error coming from the update of a RSA-signature context */
	    if(err != en_gciResult_Ok)
	    {
	        printf("GCI Error in gciSignUpdate: RSA");
	    }

	    /* Generation of the signature */
	    err = gciSignGenFinish(rsaCtxID, a_signature, &signLen);

	    /* Error coming from the generation of a RSA-signature */
	    if(err != en_gciResult_Ok)
	    {
	        printf("GCI Error in gciSignGenFinish: RSA");
	    }

	    else
	    {
	        printf("GCI Info: Signature = ");
	        for(i = 0; i < signLen; i++)
	        {
	            printf("%d", a_signature[i]);
	        }
	    }

	    /* Delete the context */
	    gciCtxRelease(rsaCtxID);

	}
\end{lstlisting}
