\chapter{Context management}

%%%%%%%%%%%%%%%%%%%%%%%%%%%%%%%%%%%%%%%%%%%%%%%%%%%%%%%%%%%%
%					DEFINITION							   %
%%%%%%%%%%%%%%%%%%%%%%%%%%%%%%%%%%%%%%%%%%%%%%%%%%%%%%%%%%%%
\section{Definition}

The contexts represent the state of stateful algorithms. It allows to avoid the
hidden states in the interface, which mean that what is entering in the context
as configuration for an algorithm will be used for the calculation and no
parameters will be statically written in the functions of the interface.

%%%%%%%%%%%%%%%%%%%%%%%%%%%%%%%%%%%%%%%%%%%%%%%%%%%%%%%%%%%%
%					CREATE CONTEXT						   %
%%%%%%%%%%%%%%%%%%%%%%%%%%%%%%%%%%%%%%%%%%%%%%%%%%%%%%%%%%%%

\section{Create a context}
The princip of context is available for the following algorithms:
\begin{itemize}
  \item Hash
  \item Signature
  \item Symmetric cipher
  \item Asymmetric cipher
  \item Diffie-Hellman
\end{itemize}


\newpage

%%%%%%%%%%%%%%%%%%%%%%%%%%%%%%%%%%%%%%%%%%%%%%%%%%%%%%%%%%%%
%					HASH CONTEXT						   %
%%%%%%%%%%%%%%%%%%%%%%%%%%%%%%%%%%%%%%%%%%%%%%%%%%%%%%%%%%%%


\subsection{Hash context}
\label{hashCtx}

Prototype:

\begin{lstlisting}
/**
 * \fn							en_gciResult_t gciHashNewCtx( en_gciHashAlgo_t hashAlgo, GciCtxId_t* p_ctxID )
 * \brief						Create a new hash context and become an ID of it
 * \param [in]  hashAlgo 		Algorithm of the hash context
 * \param [out] p_ctxID			Pointer to the context's ID
 * @return						en_gciResult_Ok on success
 * @return						en_gciResult_Err on error
 */
en_gciResult_t gciHashNewCtx( en_gciHashAlgo_t hashAlgo, GciCtxId_t* p_ctxID );
\end{lstlisting}

The context is only needed to save the Hash algorithm.
For more informations of how to use a Hash context, see chapter \ref{hashfx}.

\newpage

%%%%%%%%%%%%%%%%%%%%%%%%%%%%%%%%%%%%%%%%%%%%%%%%%%%%%%%%%%%%
%					SIGN GEN CONTEXT					   %
%%%%%%%%%%%%%%%%%%%%%%%%%%%%%%%%%%%%%%%%%%%%%%%%%%%%%%%%%%%%

\subsection{Signature context (to generate a signature) }

Prototype:
\begin{lstlisting}

/**
 * \fn							en_gciResult_t gciSignGenNewCtx( const st_gciSignConfig_t* p_signConfig, GciKeyId_t keyID, GciCtxId_t* p_ctxID )
 * \brief						Create a new signature context and become an ID of it
 * \param [in]  p_signConfig	Pointer to the configuration of the signature
 * \param [in]  keyID			Key's ID
 * \param [out] p_ctxID			Pointer to the context's ID
 * @return						en_gciResult_Ok on success
 * @return						en_gciResult_Err on error
 */
en_gciResult_t gciSignGenNewCtx( const st_gciSignConfig_t* p_signConfig, GciKeyId_t keyID, GciCtxId_t* p_ctxID );

\end{lstlisting}

This context is used to save several different configuration to generate:
\begin{itemize}
  \item RSA signature
  \item DSA signature
  \item ECDSA signature
  \item Cipher Message Authentication Code (CMAC)
  \item Hash Message Authentication Code (HMAC)
\end{itemize}

Only one configuration is possible for one context. 
\newline
For more details about each configuration listed above see chapter
\ref{signature}

\newpage

%%%%%%%%%%%%%%%%%%%%%%%%%%%%%%%%%%%%%%%%%%%%%%%%%%%%%%%%%%%%
%					SIGN VFY CONTEXT					   %
%%%%%%%%%%%%%%%%%%%%%%%%%%%%%%%%%%%%%%%%%%%%%%%%%%%%%%%%%%%%

\subsection{Signature context (to verify a signature)}

Prototype:
\begin{lstlisting}

/**
 * \fn							en_gciResult_t gciSignVerifyNewCtx( const st_gciSignConfig_t* p_signConfig, GciKeyId_t keyID, GciCtxId_t* p_ctxID )
 * \brief						Create a new signature context and become an ID of it
 * \param [in]  p_signConfig	Pointer to the configuration of the signature
 * \param [in]  keyID			Key's ID
 * \param [out] p_ctxID			Pointer to the context's ID
 * @return						en_gciResult_Ok on success
 * @return						en_gciResult_Err on error
 */
en_gciResult_t gciSignVerifyNewCtx( const st_gciSignConfig_t* p_signConfig, GciKeyId_t keyID, GciCtxId_t* p_ctxID );

\end{lstlisting}

This context is used to save several different configuration to verify:
\begin{itemize}
  \item RSA signature
  \item DSA signature
  \item ECDSA signature
  \item Cipher Message Authentication Code (CMAC)
  \item Hash Message Authentication Code (HMAC)
\end{itemize}

Only one configuration is possible for one context. 
\newline
For more details about each configuration listed above see chapter
\ref{signature}

\newpage

%%%%%%%%%%%%%%%%%%%%%%%%%%%%%%%%%%%%%%%%%%%%%%%%%%%%%%%%%%%%
%					CIPHER CONTEXT						   %
%%%%%%%%%%%%%%%%%%%%%%%%%%%%%%%%%%%%%%%%%%%%%%%%%%%%%%%%%%%%

\subsection{Cipher context}

Prototype:
\begin{lstlisting}

/**
 * \fn							en_gciResult_t gciCipherNewCtx( const st_gciCipherConfig_t* p_ciphConfig, GciKeyId_t keyID, GciCtxId_t* p_ctxID )
 * \brief						Create a new symmetric cipher context
 * \param [in]	p_ciphConfig	Pointer to the configuration of the symmetric cipher
 * \param [in]  keyID			Key's ID
 * \param [out] p_ctxID			Pointer to the context's ID
 * @return						en_gciResult_Ok on success
 * @return						en_gciResult_Err on error
 */
en_gciResult_t gciCipherNewCtx( const st_gciCipherConfig_t* p_ciphConfig, GciKeyId_t keyID, GciCtxId_t* p_ctxID );

\end{lstlisting}

This context is use to encrypt a plaintext or decrypt a ciphertext.

The cipher algorithm which could be used are:
\begin{itemize}
  \item Symmetric stream cipher RC4
  \item Symmetric block cipher AES
  \item Symmetric block cipher DES
  \item Symmetric block cipher 3DES
  \item Asymmetric cipher RSA
\end{itemize}

For more details of each configuration see chapter \ref{cipher}

\newpage

%%%%%%%%%%%%%%%%%%%%%%%%%%%%%%%%%%%%%%%%%%%%%%%%%%%%%%%%%%%%
%					DH CONTEXT							   %
%%%%%%%%%%%%%%%%%%%%%%%%%%%%%%%%%%%%%%%%%%%%%%%%%%%%%%%%%%%%

\subsection{Diffie-Hellmann context}

Prototype:
\begin{lstlisting}

/**
 * \fn							en_gciResult_t gciDhNewCtx( const st_gciDhConfig_t* p_dhConfig, GciCtxId_t* p_ctxID )
 * \brief						Create a new Diffie-Hellman context
 * \param [in]  p_dhConfig		Pointer to the configuration of the Diffie-Hellman
 * \param [out] p_ctxID			Pointer to the context's ID
 * @return						en_gciResult_Ok on success
 * @return						en_gciResult_Err on error
 */
en_gciResult_t gciDhNewCtx( const st_gciDhConfig_t* p_dhConfig, GciCtxId_t* p_ctxID );

\end{lstlisting}

This context is used to created :
\begin{itemize}
  \item Diffie-Hellman
  \item Elliptic Curve Diffie-Hellman
  \item Shared secret for Diffie-Hellman
  \item Shared secret for Elliptic Curve
Diffie-Hellman
\end{itemize}

Furthermore, the private key generated must be saved internally. 

For more information about the configuration and the generation of the
Diffie-Hellman/Elliptic Curve Diffie-Hellman key pair see chapter \ref{dhKeys}

For more informations about the generation of the shared secret see
chapter \ref{dhSec}

\newpage

%%%%%%%%%%%%%%%%%%%%%%%%%%%%%%%%%%%%%%%%%%%%%%%%%%%%%%%%%%%%
%					CLONE CONTEXT						   %
%%%%%%%%%%%%%%%%%%%%%%%%%%%%%%%%%%%%%%%%%%%%%%%%%%%%%%%%%%%%

\section{Clone an existing context}

One of the inconvenient of the interface comes from the finish part, where the
last calculation is done.
For the hash algorithm and the signature algorithm, when the digest/signature is
calculated, no more updates could be done with the last configuration and
with the previous updates.

The solution of this problem is the clone of the context.

When the digest/signature has to be calculated but the configuration and the
previous updates should be kept, the clone of the hash/signature context allows
to copy the configuration and the previous updates. Two contexts are identical
but one is use for the calculation of the digest/signature and the other one for
futur updates.

\subsection{Hash context}
Prototype:
\begin{lstlisting}

/*!
 * \fn 							en_gciResult_t gciHashCtxClone( GciCtxId_t idSrc, GciCtxId_t* p_idDest )
 * \brief						Clone a context
 * \param [in]  idSrc			The context which will be cloned
 * \param [out] p_idDest		Pointer to the context ID where the source context is cloned
 * @return						en_gciResult_Ok on success
 * @return						en_gciResult_Err on error
 */
en_gciResult_t gciHashCtxClone( GciCtxId_t idSrc, GciCtxId_t* p_idDest );

\end{lstlisting}

\subsection{Both Signature context}
Prototype:
\begin{lstlisting}

/*!
 * \fn 							en_gciResult_t gciSignCtxClone( GciCtxId_t idSrc, GciCtxId_t* p_idDest )
 * \brief						Clone a context
 * \param [in]  idSrc			The context which will be cloned
 * \param [out] p_idDest		Pointer to the context ID where the source context is cloned
 * @return						en_gciResult_Ok on success
 * @return						en_gciResult_Err on error
 */
en_gciResult_t gciSignCtxClone( GciCtxId_t idSrc, GciCtxId_t* p_idDest );

\end{lstlisting}

\newpage

%%%%%%%%%%%%%%%%%%%%%%%%%%%%%%%%%%%%%%%%%%%%%%%%%%%%%%%%%%%%
%					DELETE CONTEXT						   %
%%%%%%%%%%%%%%%%%%%%%%%%%%%%%%%%%%%%%%%%%%%%%%%%%%%%%%%%%%%%
\section{Delete an existing context}
\label{delCtx}

When the context is not needed anymore, it can be removed and be used for an
other configuration, which can be completely different as the previous one.

Prototype:
\begin{lstlisting}

/*!
 * \fn 							en_gciResult_t gciCtxRelease( GciCtxId_t ctxID )
 * \brief						Release a context
 * \param [in] ctxID			Context's ID
 * @return						en_gciResult_Ok on success
 * @return						en_gciResult_Err on error
 */
en_gciResult_t gciCtxRelease( GciCtxId_t ctxID );

\end{lstlisting}
