\chapter{Key management}
\label{keyManag}

The hardware-based cryptographic modules use a key management. This institute
wants to use this key management, that's why the interface shall have a key
management too.

\section{Save a key and get an ID}

There is two possibilities about the value of the given ID. Either the ID is
previously initialized to -1, so will the smallest value free for an ID
returned, or the ID can be initialized to a value greater or equal to 0 to be
the same as a context ID for example, so if this value is free, the key will be
saved at this specified ID.

\begin{lstlisting}
en_gciResult_t gciKeyPut( const st_gciKey_t* p_key, GciKeyId_t* p_keyID )
\end{lstlisting}

\begin{center}

\begin{tabular}{| c | *{3}{c|}}
 \hline
 Direction 	& Type 				& Parameter 			& Definition \\
 \Gline
 Input 	   	& st\_gciKey\_t*	& p\_key				& Pointer to the key to save \\
 \hline
 Output	   	& GciKeyId\_t*		& p\_keyID				& Pointer to the ID of the saved key \\
 \hline
\end{tabular}
\captionof{table}{Parameters for to put a key}
\label{tab:key_put}

\end{center}

\section{Get of a saved key with its ID}

\begin{lstlisting}
en_gciResult_t gciKeyGet( GciKeyId_t keyID, st_gciKey_t* p_key )
\end{lstlisting}

\begin{center}

\begin{tabular}{| c | *{3}{c|}}
 \hline
 Direction 	& Type 				& Parameter 			& Definition \\
 \Gline
 Input 	   	& GciKeyId\_t		& p\_key				& ID of the key to get \\
 \hline
 Output	   	& st\_gciKey\_t*	& p\_keyID				& Pointer to the saved key \\
 \hline
\end{tabular}
\captionof{table}{Parameters for to get a key}
\label{tab:key_get}

\end{center}

\section{Delete a key}

\begin{lstlisting}
en_gciResult_t gciKeyDelete( GciKeyId_t keyID  )
\end{lstlisting}

\begin{center}

\begin{tabular}{| c | *{3}{c|}}
 \hline
 Direction 	& Type 				& Parameter 			& Definition \\
 \Gline
 Input 	   	& GciKeyId\_t		& p\_key				& ID of the key to delete \\
 \hline
\end{tabular}
\captionof{table}{Parameters for to delete a key}
\label{tab:key_del}

\end{center}

