\chapter{Cipher algorithms}
\label{cipher}

\section{Configuration of cipher algorithms}

\begin{center}

\begin{tabular}{| c | c|}
 \hline
Algorithm				& Parameter \\
\Gline
RC4						& en\_gciCipherAlgo\_Rc4 \\
\hline
DES						& en\_gciCipherAlgo\_Des \\
\hline
3DES					& en\_gciCipherAlgo\_3des \\
\hline
AES						& en\_gciCipherAlgo\_Aes \\
\hline
RSA						& en\_gciCipherAlgo\_Rsa \\
\hline
\end{tabular}
\captionof{table}{Cipher algorithms (en\_gciCipherAlgo\_t) }
\label{tab:cipher_algo}

\end{center}



\begin{center}

\begin{tabular}{| c | c | c |}
\hline
Parameter		& Type \\				
\Gline
algo			& en\_gciCipherAlgo\_t \\
\hline
blockMode		& en\_gciBlockMode\_t \\
(If DES, 3DES or AES is used as algorithm)	& \\
\hline
padding			& en\_gciPadding\_t \\	
(If RSA is used as algorithm)& \\		
\hline
iv				& st\_gciBuffer\_t \\
(If DES, 3DES or AES is used as algorithm)	& \\
\hline	


\end{tabular}
\captionof{table}{Configuration of cipher algorithms
(st\_gciCipherConfig\_t))}
\label{tab:cipher_conf}

\end{center}

\section{Prototypes}

\subsection{Creation of a context}

\begin{lstlisting}
en_gciResult_t gciCipherNewCtx( const st_gciCipherConfig_t* p_ciphConfig,
GciKeyId_t keyID, GciCtxId_t* p_ctxID )
\end{lstlisting}

\begin{center}

\begin{tabular}{| c | *{3}{c|}}
 \hline
 Direction 	& Type 						& Parameter 			& Definition \\
 \Gline
 Input 	   	& st\_gciCipherConfig\_t*	& p\_ciphConfig			& Pointer to the
 configuration of the cipher
 \\
 \hline
 Input	   	& GciKeyId\_t			 	& keyID					& ID of the key uses to encrypt/decrypt
 \\
 \hline
Output		& GciCtxId\_t* 				& p\_ctxID				& Pointer to the
context's ID \\
\hline
\end{tabular}
\captionof{table}{Parameters for the creation of a cipher context}
\label{tab:ciph_ctx}

\end{center}

For a symmetric cipher, the key, represented here by the ID, should be the same
for the two parts of the communication for the encryption and the decryption of data.

For an asymmetric cipher, the key, reprensented here by the ID, must be the
public key of the key pair for the encryption and the private key for the
decryption of data.

\subsubsection*{Symmetric stream cipher}

\begin{center}

\begin{tabular}{| c | *{2}{c|}}
 \hline
 Parameter 		& Configuration				& Prohibition(s) \\
 \Gline
 algo 	   		& en\_gciSignAlgo\_Rc4 		& - \\
\hline
 blockMode		& en\_gciBlockMode\_None	& - \\					
 \hline
 padding		& en\_gciPadding\_None 		& - \\
 \hline
 iv				& NULL						& - \\
 \hline
\end{tabular}
\captionof{table}{Configuration of symmetric stream ciphers
(st\_gciCipherConfig\_t)}
\label{tab:ciph_stream}

\end{center}

\subsubsection*{Symmetric block cipher}

\begin{center}

\begin{tabular}{| c | *{2}{c|}}
 \hline
 Parameter 		& Configuration				& Prohibition(s) \\
 \Gline
 algo 	   		& en\_gciSignAlgo\_Des 		& en\_gciCipherAlgo\_Invalid \\
 				& en\_gciSignAlgo\_3Des		& en\_gciCipherAlgo\_None \\
 				& en\_gciSignAlgo\_Aes		& en\_gciCipherAlgo\_Rc4 \\
 				&							& en\_gciCipherAlgo\_Rsa \\
\hline
 blockMode		& en\_gciBlockMode\_t 		& en\_gciBlockMode\_Invalid \\
 				&							& en\_gciBlockMode\_None \\					
 \hline
 padding		& en\_gciPadding\_None 		& - \\
 \hline
 iv				& st\_gciBuffer\_t			& less or equal than 0 \\
 \hline
\end{tabular}
\captionof{table}{Configuration of symmetric block ciphers
(st\_gciCipherConfig\_t)}
\label{tab:ciph_block}

\end{center}

\subsubsection*{Asymmetric cipher}

\begin{center}

\begin{tabular}{| c | *{2}{c|}}
 \hline
 Parameter 		& Configuration				& Prohibition(s) \\
 \Gline
 algo 	   		& en\_gciSignAlgo\_Rsa 		& - \\
\hline
 blockMode		& en\_gciBlockMode\_None	& - \\					
 \hline
 padding		& en\_gciPadding\_t			& en\_gciPadding\_Invalid \\
 				&							& en\_gciPadding\_None \\
 \hline
 iv				& NULL						& - \\
 \hline
\end{tabular}
\captionof{table}{Configuration of asymmetric ciphers
(st\_gciCipherConfig\_t)}
\label{tab:ciph_asym}

\end{center}

\subsection{Encryption a plaintext}

\begin{lstlisting}
en_gciResult_t gciCipherEncrypt( GciCtxId_t ctxId, const uint8_t* p_plaintxt,
size_t pltxtLen, uint8_t* p_ciphtxt, size_t* p_cptxtLen )
\end{lstlisting}

\begin{center}

\begin{tabular}{| c | *{3}{c|}}
 \hline
 Direction 	& Type 				& Parameter 			& Definition \\
 \Gline
 Input 	   	& GciCtxId\_t		& ctxID					& Context's ID \\
 \hline
 Input	   	& uint8\_t*			& p\_plaintext			& Pointer to the plaintext to encrypt \\
 \hline
 Input		& size\_t			& pltxtLen				& Length of the plaintext to encrypt \\
 \hline
 Output		& uint8\_t*			& p\_ciphtxt			& Pointer to the ciphertext \\
 \hline
 Output		& size\_t*			& p\_cptxtLen			& Pointer to the length of the ciphertext
 \\
 \hline
\end{tabular}
\captionof{table}{Parameters for the encryption of a plaintext}
\label{tab:ciph_enc}

\end{center}

\subsection{Decryption a ciphertext}

\begin{lstlisting}
en_gciResult_t gciCipherDecrypt( GciCtxId_t ctxId, const uint8_t* p_ciphtxt,
size_t cptxtLen, uint8_t* p_plaintxt, size_t* p_pltxtLen );
\end{lstlisting}

\begin{center}

\begin{tabular}{| c | *{3}{c|}}
 \hline
 Direction 	& Type 				& Parameter 			& Definition \\
 \Gline
 Input 	   	& GciCtxId\_t		& ctxID					& Context's ID \\
 \hline
 Input	   	& uint8\_t*			& p\_ciphtxt			& Pointer to the ciphertext to decrypt
 \\
 \hline
 Input		& size\_t			& cptxtLen				& Length of the ciphertext to decrypt
 \\
 \hline
 Output		& uint8\_t*			&  p\_plaintext			& Pointer to the plaintext \\
 \hline
 Output		& size\_t*			& p\_pltxtLen			& Pointer to the length of the plaintext
 \\
 \hline
\end{tabular}
\captionof{table}{Parameters for the decryption of a ciphertext}
\label{tab:ciph_dec}

\end{center}


\section{Step to encrypt and decrypt data}

For this example the keys have already been added previously to the interface 
and an ID returned.

\begin{lstlisting}

#include <stdio.h>
#include <stdlib.h>
#include <string.h>
#include "crypto_iface.h"
int main(int argc , char *argv[])
{

    /* Error management */
    en_gciResult_t err;

    int i;

    uint8_t a_pltxt[20] = {"Data to encrypt"};
    size_t pltxtLen = strlen(a_pltxt);

    uint8_t a_ciphtxt[50];
    memset(a_ciphtxt, 0, 50);
    size_t ciphtxtLen = 0;

    /* Configuration of a RSA cipher uses for encryption */
    st_gciCipherConfig_t rsaConfEnc = {.algo = en_gciCipherAlgo_Rsa,
        .blockMode = en_gciBlockMode_None,
        .padding = en_gciPadding_Pkcs1_Emsa,
        .iv = NULL
    };

    /* Context's ID */
    GciCtxId_t rsaEncCtxID = -1;
    GciCtxId_t rsaDecCtxID = -1;

    /* Creation of the cipher context */
    err = gciCipherNewCtx(&rsaConfEnc, rsaPubKeyID, &rsaEncCtxID);

    /* Error coming from the creation of a RSA cipher context */
    if(err != en_gciResult_Ok)
    {
        printf("GCI Error in gciCipherNewCtx: RSA");
    }

    /* Encrytion of the plaintext */
    err = gciCipherEncrypt(rsaEncCtxID, a_pltxt, pltxtLen, a_ciphtxt, &ciphtxtLen);

    /* Error coming from the encryption with a RSA cipher context */
    if(err != en_gciResult_Ok)
    {
        printf("GCI Error in gciCipherEncrypt: RSA");
    }

    else
    {

        printf("Ciphertext:");

        for(i=0; i<ciphtxtLen; i++)
        {
            printf("%d", a_ciphtxt[i]);
        }
    }


    /* Configuration of a RSA cipher uses for decryption */
    st_gciCipherConfig_t rsaConfDec = {.algo = en_gciCipherAlgo_Rsa,
        .blockMode = en_gciBlockMode_None,
        .padding = en_gciPadding_Pkcs1_Emsa,
        .iv = NULL
    };

    memset(a_pltxt, 0, 50);
    pltxtLen = 0;


    /* Creation of the cipher context */
    err = gciCipherNewCtx(&rsaConfDec, rsaPrivKeyID, &rsaDecCtxID);

    /* Error coming from the creation of a RSA cipher context */
    if(err != en_gciResult_Ok)
    {
        printf("GCI Error in gciCipherNewCtx: RSA");
    }

    /* Decryption of the ciphertext */
    err = gciCipherDecrypt(rsaDecCtxID, a_ciphtxt, ciphtxtLen, a_pltxt, &pltxtLen);

    /* Error coming from the decryption with a RSA cipher context */
    if(err != en_gciResult_Ok)
    {
        printf("GCI Error in gciCipherDecrypt: RSA");
    }

    else
    {

        printf("Plaintext:");

        for(i=0; i<pltxtLen; i++)
        {
            printf("%d", a_pltxt[i]);
        }
    }

}
\end{lstlisting}
