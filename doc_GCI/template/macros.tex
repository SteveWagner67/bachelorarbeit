%---------------------------------------------------------------------------------------------------
% Settings
%---------------------------------------------------------------------------------------------------
\newcommand{\mypapersize}{A4}
\newcommand{\mylaterality}{oneside}
%% "oneside" or "twoside"
\newcommand{\mydraft}{false}
%% "true" or "false"
\newcommand{\myparskip}{half}
%% e.g., "no", "full", "half", ...
\newcommand{\myBCOR}{10mm}
\newcommand{\myfontsize}{11pt}   
\newcommand{\mylinespread}{onehalfspacing} 
%% e.g.onehalfspacing, doublespacing, singlespacing
%% Line spacing in %/100. For example 1.5 means 150% of the usual line
%% spacing. Please use with caution: 100% ("1.0") is fine because the
%% font was designed for it.
\newcommand{\mylanguage}{ngerman,american}
%% NOTE: The *last* language is the active one!
%% BibLaTeX-settings: (see biblatex reference for further description)
\newcommand{\mybiblatexstyle}{numeric}
%% e.g., "alphabetic", "authoryear", ...
%% The biblatex style which is being used for referencing. See
%% biblatex documentation for further details and more values.
%%
%% CAUTION: if you change the style, please check for (in)compatible
%%          "biblatex" package options in the file
%%          "template/preamble.tex"! For example: "alphabetic" does
%%          not have an option "dashed=..." and causes an error if it
%%          does not get removed from the list of options.

\newcommand{\mybiblatexdashed}{false}  %% "true" or "false"
%% If true: replace recurring reference authors with a dash.

\newcommand{\mybiblatexbackref}{true}  %% "true" or "false"
%% If true: create backward links from reference to citations.

\newcommand{\mybiblatexfile}{bib/bibliography.bib}
%% Name of the biblatex file that holds the references.

\newcommand{\mydispositioncolor}{0,0,0}
%% e.g., "30,103,182" (blue/turquois), "0,0,0" (black), ...
%% Color of the headings and so forth in RGB (red,green,blue) values.

\newcommand{\mycolorlinks}{false}  %% "true" or "false"
%% Enables or disables colored links (hyperref package).
\newcommand{\mytodonotesoptions}{disable}
%% e.g., "" (empty), "disable", ...
%% Options for the todonotes-package. If "disable", all todonotes will
%% be hidden (including todos).
%% ========================================================================
%%  Document metadata
%% ========================================================================
%% general metadata:
\newcommand{\myauthor}{Steve Wagner}  %% also used for PDF metadata (hyperref)
\newcommand{\myformation}{EI-3nat}
\newcommand{\mytitle}{Documentation}  %% also used for PDF metadata (hyperref)
\newcommand{\mysubject}{Generic Cryptographic Interface}  %% also used for PDF metadata (hyperref)
\newcommand{\mykeywords}{<++keywords++>}  %% also used for PDF metadata (hyperref)
%% this information is used only for generating the title page:
\newcommand{\myworktitle}{Bachelor thesis}  %% official type of work like ``Master theses''
\newcommand{\mygrade}{Bachelor of Engineering} %% title you are getting with this work like ``Master of ...''
\newcommand{\mystudy}{Electronik und Informationstechnik} %% your study like ``Arts''
\newcommand{\myuniversity}{Offenburg University of Applied Sciences} %% your
% university/school
\newcommand{\myinstitute}{Institute of reliable Embedded Systems and Communication
Electronics (ivESK)} %% affiliation
\newcommand{\myinstitutehead}{Prof. Dr. Axel Sikora} %% head of institute 
\newcommand{\mysupervisor}{Dipl.-Phys. Andreas Walz} %% your supervisor
\newcommand{\myevaluator}{myprof} %% your evaluator
\newcommand{\myhomestreet}{street} %% your home street (with house number)
\newcommand{\myhometown}{town} %% your home town
\newcommand{\myhomepostalnumber}{psn} %% your postal number of home town
\newcommand{\mysubmissionmonth}{month} %% month you are handing in
\newcommand{\mysubmissionyear}{year} %% year you are handing in
\newcommand{\mysubmissiontown}{\myhometown} %% town of handing in (or \myhometown)
%% additional information for generic_documentation title page

%---------------------------------------------------------------------------------------------------
% formating
%---------------------------------------------------------------------------------------------------

\newcommand{\clearemptydoublepage}{\clearpage\newpage\thispagestyle{empty}\cleardoublepage}

\newcommand{\CRule}{\rule{0.95\textwidth}{0.5pt}} % New command to make the lines above figure captions


%---------------------------------------------------------------------------------------------------
% fixme makro
%---------------------------------------------------------------------------------------------------

%\newcommand{\fixme}[1]{\textbf{\large FIXME: #1}}
%\newcommand{\todo}[1]{\textbf{\large TODO: #1}}
%\newcommand{\idea}[1]{\textbf{IDEA: #1 ~\\}}


%---------------------------------------------------------------------------------------------------
% names
%---------------------------------------------------------------------------------------------------

\newcommand{\Table}{Table}
\newcommand{\Tables}{Tables}
\newcommand{\Figure}{Figure}
\newcommand{\Figures}{Figures}
\newcommand{\Subfigure}{Subfigure}
\newcommand{\Section}{Section}
\newcommand{\Sections}{Sections}
\newcommand{\Chapter}{Chapter}
\newcommand{\Chapters}{Chapters}
\newcommand{\Equation}{Equation}
\newcommand{\Equations}{Equations}

\newcommand{\Appendix}{Appendix}
\newcommand{\Appendices}{Appendices}
\newcommand{\Ref}{Ref.}
\newcommand{\Refs}{Refs.}

%---------------------------------------------------------------------------------------------------
% units
%---------------------------------------------------------------------------------------------------

\newcommand{\Hz}	{\ensuremath{\mathrm{Hz}}\xspace}
\newcommand{\kHz}	{\ensuremath{\mathrm{kHz}}\xspace}
\newcommand{\MHz}	{\ensuremath{\mathrm{MHz}}\xspace}

\newcommand{\cm}{\ensuremath{\mathrm{cm}}\xspace}
\newcommand{\m}{\ensuremath{\mathrm{m}}\xspace}
\newcommand{\mm}{\ensuremath{\mathrm{mm}}\xspace}
\newcommand{\microm}{\ensuremath{\mu\mathrm{m}}\xspace}
\newcommand{\s}{\ensuremath{\mathrm{s}}\xspace}
\newcommand{\musec}{\ensuremath{\mu\mathrm{s}}\xspace}


%---------------------------------------------------------------------------------------------------
% include figures
%---------------------------------------------------------------------------------------------------
\newcommand{\myfig}[5]{
%% example:
% \myfig{}%% filename in figures folder
%       {width=0.5\textwidth,height=0.5\textheight}%% maximum width/height, aspect ratio will be kept
%       {}%% caption
%       {}%% optional (short) caption for list of figures
%       {}%% label
\begin{figure}%[htp]
  \begin{center}
     \includegraphics[keepaspectratio,#2]{figures/#1}
     \caption[#4]{#3}
     \label{#5} %% NOTE: always label *after* caption!
  \end{center}
  
\end{figure}
}

 \newlength\epaisLigne 

\newcommand\Gline{\noalign{\global\epaisLigne\arrayrulewidth\global
                              \arrayrulewidth 0.1cm}
		    \hline\noalign{\global\arrayrulewidth\epaisLigne}} 
		    