%%=====================================================================================
%%
%%       Filename:  preamble.tex
%%
%%    Description:  Thesis Template  
%%
%%        Version:  1.0
%%        Created:  25.08.2015
%%       Revision:  none
%%
%%         Author:  B.Eng. Oliver Kehret, okehret@stud.hs-offenburg.de
%%   Organization:  HS Offenburg, Offenburg, Germany
%%      Copyright:  Copyright (c) 2015, B.Eng. Oliver Kehret
%%
%%          Notes:  !!! important !!! on debian based systems you have to install
%%                  biber with sudo apt-get install biber for the bibliography or use 
%%                  bibtex instead but biber/biblatex has utf8 support per default
%%                
%%=====================================================================================
\documentclass[%
    fontsize=\myfontsize,
	paper=\mypapersize,
	parskip=\myparskip,
	DIV=calc,
	headinclude=true,		
	footinclude=true,
	open=right,
	appendixprefix=true,	% include appendix?
	bibliography=totoc,		% include an unnumbered unit of bibliography to the table of contents
	BCOR=\myBCOR,        	% binding correction (depends on how you bind
	                     	% the resulting printout.
	\mylaterality,          % alternative: twoside
    \mylanguage
]{scrbook}

%%=====================================================================================
%% General presentation
%%=====================================================================================
% full utf8 character set
\usepackage[utf8]{inputenc}
% language adaptions, change in main.tex (english, ngerman, american)
\usepackage[\mylanguage]{babel}									
% encode characters better, so that they can be copied out of .pdf in one piece
\usepackage[T1]{fontenc}    
% nicer quotes  
\usepackage[%
            autostyle,          % adapts language setting
            strict,             % turns warnings into errors 
            english=american,   % use american quotes style
]{csquotes}

\usepackage{todonotes}

%%=====================================================================================
%% Style and Embedded-Lab-look-alike
%%=====================================================================================
\usepackage[automark]{scrpage2}							% allows usage of header and footer		
\usepackage[perpage, hang]{footmisc} 						% footnote options
% perpage: Nummerierung der Fu�noten auf jeder Seite neu beginnen
% hang: Fu�note �ber mehrere zeilen richtig einr�cken

%%=====================================================================================
%% includes pictures --> possible to include them in header/footer
%%=====================================================================================
\usepackage[pdftex]{graphicx}										


%%=====================================================================================
%% Style and Embedded-Lab-look-alike Part 2
%%=====================================================================================
%%% includes some colors
\usepackage[table,usenames,dvipsnames]{xcolor}		
\definecolor{DispositionColor}{RGB}{\mydispositioncolor}							
\definecolor{light-gray}{gray}{0.95}	% define some colors
\definecolor{dark-green}{rgb}{0,0.6,0}
\renewcommand{\headfont}{\normalfont\sffamily\color{DispositionColor}}
\renewcommand{\pnumfont}{\normalfont\sffamily\color{DispositionColor}}
\addtokomafont{disposition}{\color{DispositionColor}}
\addtokomafont{caption}{\color{DispositionColor}\footnotesize}
\addtokomafont{captionlabel}{\color{DispositionColor}}
%% used for calculation of header footer etc. ...
\usepackage{calc}											
%% change page layout
% ftp://ftp.fu-berlin.de/tex/CTAN/macros/latex/contrib/geometry/geometry.pdf
\addtolength{\oddsidemargin}{-.875in}
\addtolength{\evensidemargin}{-.875in}
\addtolength{\textwidth}{1.75in}
\addtolength{\topmargin}{-.875in}
\addtolength{\textheight}{3in}  %1.75
% header and footer
\pagestyle{scrheadings}							% for customization for header and footer
\renewcommand{\chapterpagestyle}{scrheadings}	% include header and footer on chapter pages
\clearscrheadfoot

% http://ctan.uib.no/macros/latex/contrib/koma-script/doc/scrguien.pdf
% page 204
\ifoot{\vspace{-0.25cm}\mytitle~\myauthor}
\ofoot{ \vspace{-0.25cm} \thepage}
\automark{chapter}
\setheadsepline[ \textwidth + 5pt ]{1pt} % seperation line for header...
\setfootsepline[ \textwidth + 5pt ]{1pt} % ... and footer
\setkomafont{footsepline}{\color{red}} 	 % change colors of seperation lines
\setkomafont{headsepline}{\color{red}}
%%=====================================================================================
%% bibliography with biber/biblatex
%%=====================================================================================
%\usepackage[backend=biber, %% using "biber" to compile references (instead of "biblatex")
%style=\mybiblatexstyle, %% see biblatex documentation
%style=alphabetic, %% see biblatex documentation
%dashed=\mybiblatexdashed, %% do *not* replace recurring reference authors with a dash
%backref=\mybiblatexbackref, %% create backlings from references to citations
%natbib=true, %% offering natbib-compatible commands
%hyperref=true,
%sorting=none, %% using hyperref-package references
%]{biblatex}  %% remove, if using BibTeX instead of biblatex
%\addbibresource{\mybiblatexfile} %% remove, if using BibTeX instead of biblatex

%%=====================================================================================
%% Selection of useful packages
%%=====================================================================================
%%
%%=====================================================================================
%% SIUNITSX -- simplified usage of SI-units
%%=====================================================================================
\usepackage[% 
            exponent-product = \cdot % use \cdot instead * for exponent product
]{siunitx}	              
%%=====================================================================================
%% ifthen and todonotes puts to-do-notes in the printed document if you want 
%%=====================================================================================
%% used to disable todonotes package
\usepackage{ifthen}                                         
%% pre-define ifthen-boolean variables:
\newboolean{myaddcolophon}
\newboolean{myaddlistoftodos}
%
\usepackage[\mytodonotesoptions]{todonotes}
%%=====================================================================================
%% Sourcecode printing
%%=====================================================================================
\usepackage{listings}				% include source code
									% ftp://ftp.tex.ac.uk/tex-archive/macros/latex/contrib/listings/listings.pdf
\lstset{% 							% options for representation of source code
  backgroundcolor=\color{light-gray},   % choose the background color; you must add \usepackage{color} or \usepackage{xcolor}
  basicstyle=\footnotesize,        % the size of the fonts that are used for the code
  breakatwhitespace=false,         % sets if automatic breaks should only happen at whitespace
  breaklines=true,                 % sets automatic line breaking
  captionpos=b,                    % ses the caption-position to bottom
  commentstyle=\color{dark-green}, % comment style
  deletekeywords={...},            % if you want to delete keywords from the given language
  escapeinside={\%*}{*)},          % if you want to add LaTeX within your code
  extendedchars=true,              % lets you use non-ASCII characters; for 8-bits encodings only, does not work with UTF-8
  frame=lines,                     % adds a frame around the code
  keepspaces=true,                 % keeps spaces in text, useful for keeping indentation of code (possibly needs columns=flexible)
  keywordstyle=\color{blue},       % keyword style
  language=C,                      % the language of the code
  morekeywords={*,...},            % if you want to add more keywords to the set
  numbers=left,                    % where to put the line-numbers; possible values are (none, left, right)
  numbersep=8pt,                   % how far the line-numbers are from the code
  numberstyle=\tiny\color{gray},   % the style that is used for the line-numbers
  rulecolor=\color{black},         % if not set, the frame-color may be changed on line-breaks within not-black text (e.g. comments (green here))
  showspaces=false,                % show spaces everywhere adding particular underscores; it overrides 'showstringspaces'
  showstringspaces=false,          % underline spaces within strings only
  showtabs=false,                  % show tabs within strings adding particular underscores
  stepnumber=1,                    % the step between two line-numbers. If it's 1, each line will be numbered
  stringstyle=\color{blue},        % string literal style
  tabsize=2,                       % sets default tabsize to 2 spaces
  title=\lstname,                  % show the filename of files included with \lstinputlisting; also try caption instead of title
  %numberbychapter=false
  %keywordstyle=\bfseries\color{green!40!black},
  %commentstyle=\itshape\color{purple!40!black},
  %identifierstyle=\color{blue},
  %stringstyle=\color{orange}
}

\renewcommand\lstlistingname{Quelltext}	% change caption of listings
%%=====================================================================================
%% Math 
%%=====================================================================================
\usepackage{amssymb,amstext} %% predefiened symbols e.g. \nparallel
\usepackage[%
            fleqn,%equations aligned in a fixed distance from the left
            tbtags, %where the equation number is placed here bottom or top
]{mathtools} %% loads amsmath package

%%=====================================================================================
%% Tables, figures etc.
%%=====================================================================================
%
%% nice rule's for tables try \toprule \midrule \bottomrule  
\usepackage{booktabs}
%% set width of table and more
\usepackage{tabularx}										% creates tables
%
%% rotate tables and figures
\usepackage{rotating}
%
%% define caption style
\addtokomafont{caption}{\small} 	% small captions
\usepackage[font=small, width=0.9\textwidth, format=plain, labelfont=bf]{caption}
%
%%
\usepackage{subfigure}

%%=====================================================================================
%% some utility stuff
%%=====================================================================================
%
\usepackage[]{acronym}				% for usage of abbreviations
%
% improved typographical settings
\usepackage[%
    protrusion=true, %
    factor=900       %
]{microtype}
%
%% switch of extra space after punctuation
\frenchspacing 
%
%% switches to Palatino with small caps and old style figures
\usepackage[%
sc,%
osf,%
]{mathpazo}
%
%% customize item look
\usepackage{enumitem}
%% kills space between items
%\setlist{noitemsetup}
%doc% For additional special characters available by \verb#\ding{}#
\usepackage{pifont}  %% Sonderzeichen fuer Titelseite \ding{}
%
%doc% This package is required for intelligent spacing after commands
\usepackage{xspace}
%
%
%doc% This package offers strikethrough command \verb+\sout{foobar}+.
\usepackage[normalem]{ulem}
%
%
%doc% Create framed, shaded, or differently highlighted regions that can 
%doc% break across pages.  The environments defined are 
%doc% \begin{itemize}
%doc%   \item framed: ordinary frame box (\verb+\fbox+) with edge at margin
%doc%   \item shaded: shaded background (\verb+\colorbox+) bleeding into margin
%doc%   \item snugshade: similar
%doc%   \item leftbar: thick vertical line in left margin
%doc% \end{itemize}
\usepackage{framed}
%
%doc% For example on title pages you might want to have a logo on the upper right corner of
%doc% the first page (only). The package \texttt{eso-pic} is able to place things on absolute
%doc% and relative positions on the whole page.
\usepackage{eso-pic} %%
%% for what ????
%\usepackage{lastpage}										% get total number of pages
%

%%=====================================================================================
%% drawing tikz
%%=====================================================================================
%
%% best way is to draw in different file and include in maindocument as it really slows down


%%%=====================================================================================
%% Own Colors for header, captions etc.
%%=====================================================================================
%





%% prevent club & widow penalty
\clubpenalty10000
\widowpenalty10000
\displaywidowpenalty10000

%%=====================================================================================
%% pdfcompresslevel from 0 to 10; std is fine 
%%=====================================================================================
\pdfcompresslevel=9 
%%=====================================================================================
%% Hyperref should always be the last package added -- 
%%=====================================================================================
\usepackage[%							% enables typesettings for hyperlinks
	                    				% http://en.wikibooks.org/wiki/LaTeX/Hyperlinks	
	pdftitle={\mysubject},%
	pdfauthor = {\myauthor},%
	pdfsubject = {\mysubject},%
    %colorlinks={\mycolorlinks},					% removes color frames 
    colorlinks=false,
    pdfcreator={pdfTex},%
    pdfkeywords={\mykeywords},%
    pdftex = true,%
    backref,%
    pagebackref=false, % creates backward references too
    bookmarks=false, %
    bookmarksopen=false, % when starting with AcrobatReader, the Bookmarkcolumn is opened
    pdfpagemode=UseOutlines,% None, UseOutlines, UseThumbs, FullScreen
    plainpages=false, % correct, if pdflatex complains: ``destination with same identifier already exists''
    %% colors: https://secure.wikimedia.org/wikibooks/en/wiki/LaTeX/Colors
	hidelinks,							% removes color and porder
	%breaklinks=true,
]{hyperref}								% should be the last package to be inlcuded!

\usepackage{lipsum}
\usepackage{titlesec}

\titlespacing\chapter{0pt}{12pt plus 4pt minus 2pt}{0pt plus 2pt minus 2pt}
\titlespacing\section{0pt}{12pt plus 4pt minus 2pt}{0pt plus 2pt minus 2pt}
\titlespacing\subsection{0pt}{12pt plus 4pt minus 2pt}{0pt plus 2pt minus 2pt}
\titlespacing\subsubsection{0pt}{12pt plus 4pt minus 2pt}{0pt plus 2pt minus 2pt}



%\usepackage[colorlinks=false, urlcolor=red, breaklinks, pagebackref, citebordercolor={0 0 0}, filebordercolor={0 0 0}, linkbordercolor={0 0 0}, pagebordercolor={0 0 0}, runbordercolor={0 0 0}, urlbordercolor={0 0 0}, pdfborder={0 0 0}]{hyperref}



