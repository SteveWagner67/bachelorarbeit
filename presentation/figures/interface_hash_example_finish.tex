%---------------------------------------------------------------------------------------------------
% Settings
%---------------------------------------------------------------------------------------------------
\newcommand{\mypapersize}{A4}
\newcommand{\mylaterality}{oneside}
%% "oneside" or "twoside"
\newcommand{\mydraft}{false}
%% "true" or "false"
\newcommand{\myparskip}{half}
%% e.g., "no", "full", "half", ...
\newcommand{\myBCOR}{10mm}
\newcommand{\myfontsize}{11pt}   
\newcommand{\mylinespread}{onehalfspacing} 
%% e.g.onehalfspacing, doublespacing, singlespacing
%% Line spacing in %/100. For example 1.5 means 150% of the usual line
%% spacing. Please use with caution: 100% ("1.0") is fine because the
%% font was designed for it.
\newcommand{\mylanguage}{ngerman,american}
%% NOTE: The *last* language is the active one!
%% BibLaTeX-settings: (see biblatex reference for further description)
\newcommand{\mybiblatexstyle}{numeric}
%% e.g., "alphabetic", "authoryear", ...
%% The biblatex style which is being used for referencing. See
%% biblatex documentation for further details and more values.
%%
%% CAUTION: if you change the style, please check for (in)compatible
%%          "biblatex" package options in the file
%%          "template/preamble.tex"! For example: "alphabetic" does
%%          not have an option "dashed=..." and causes an error if it
%%          does not get removed from the list of options.

\newcommand{\mybiblatexdashed}{false}  %% "true" or "false"
%% If true: replace recurring reference authors with a dash.

\newcommand{\mybiblatexbackref}{true}  %% "true" or "false"
%% If true: create backward links from reference to citations.

\newcommand{\mybiblatexfile}{bib/bibliography.bib}
%% Name of the biblatex file that holds the references.

\newcommand{\mydispositioncolor}{0,0,0}
%% e.g., "30,103,182" (blue/turquois), "0,0,0" (black), ...
%% Color of the headings and so forth in RGB (red,green,blue) values.

\newcommand{\mycolorlinks}{false}  %% "true" or "false"
%% Enables or disables colored links (hyperref package).
\newcommand{\mytodonotesoptions}{disable}
%% e.g., "" (empty), "disable", ...
%% Options for the todonotes-package. If "disable", all todonotes will
%% be hidden (including todos).
%% ========================================================================
%%  Document metadata
%% ========================================================================
%% general metadata:
\newcommand{\myauthor}{Steve Wagner}  %% also used for PDF metadata
% (hyperref)
\newcommand{\myformation}{EI-3nat}
\newcommand{\mytitle}{Bachelor Thesis}  %% also used for PDF metadata (hyperref)
\newcommand{\mysubject}{Extension and Integration of an Abstract Interface to Cryptography Providers}  %% also used for PDF metadata (hyperref)
\newcommand{\mykeywords}{<++keywords++>}  %% also used for PDF metadata (hyperref)
%% this information is used only for generating the title page:
\newcommand{\myworktitle}{Bachelor thesis}  %% official type of work like ``Master theses''
\newcommand{\mygrade}{Bachelor of Engineering} %% title you are getting with this work like ``Master of ...''
\newcommand{\mystudy}{Electronik und Informationstechnik} %% your study like ``Arts''
\newcommand{\myuniversity}{Offenburg University of Applied Sciences} %% your
% university/school
\newcommand{\myinstitute}{Institute of reliable Embedded Systems and
Communication Electronics (ivESK)}
%% affiliation
\newcommand{\myinstitutehead}{Prof. Dr. Axel Sikora} %% head of institute 
\newcommand{\mysupervisor}{Dipl.-Phys. Andreas Walz} %% your supervisor
\newcommand{\myevaluator}{myprof} %% your evaluator
\newcommand{\myhomestreet}{street} %% your home street (with house number)
\newcommand{\myhometown}{town} %% your home town
\newcommand{\myhomepostalnumber}{psn} %% your postal number of home town
\newcommand{\mysubmissionmonth}{month} %% month you are handing in
\newcommand{\mysubmissionyear}{year} %% year you are handing in
\newcommand{\mysubmissiontown}{\myhometown} %% town of handing in (or \myhometown)
%% additional information for generic_documentation title page

%---------------------------------------------------------------------------------------------------
% formating
%---------------------------------------------------------------------------------------------------

\newcommand{\clearemptydoublepage}{\clearpage\newpage\thispagestyle{empty}\cleardoublepage}

\newcommand{\CRule}{\rule{0.95\textwidth}{0.5pt}} % New command to make the lines above figure captions


%---------------------------------------------------------------------------------------------------
% fixme makro
%---------------------------------------------------------------------------------------------------

%\newcommand{\fixme}[1]{\textbf{\large FIXME: #1}}
%\newcommand{\todo}[1]{\textbf{\large TODO: #1}}
%\newcommand{\idea}[1]{\textbf{IDEA: #1 ~\\}}


%---------------------------------------------------------------------------------------------------
% names
%---------------------------------------------------------------------------------------------------

\newcommand{\gci}{Generic Cryptographic Interface (GCI)\xspace}
\newcommand{\embtls}{emb::TLS\xspace}
\newcommand{\tomcrypt}{LibTomCrypt\xspace}
\newcommand{\vaultic}{VaultIC\num{460}\xspace}
\newcommand{\Table}{Table}
\newcommand{\Tables}{Tables}
\newcommand{\Figure}{Figure}
\newcommand{\Figures}{Figures}
\newcommand{\Subfigure}{Subfigure}
\newcommand{\Section}{Section}
\newcommand{\Sections}{Sections}
\newcommand{\Chapter}{Chapter}
\newcommand{\Chapters}{Chapters}
\newcommand{\Equation}{Equation}
\newcommand{\Equations}{Equations}

\newcommand{\Appendix}{Appendix}
\newcommand{\Appendices}{Appendices}
\newcommand{\Ref}{Ref.}
\newcommand{\Refs}{Refs.}

%---------------------------------------------------------------------------------------------------
% units
%---------------------------------------------------------------------------------------------------

\newcommand{\Hz}	{\ensuremath{\mathrm{Hz}}\xspace}
\newcommand{\kHz}	{\ensuremath{\mathrm{kHz}}\xspace}
\newcommand{\MHz}	{\ensuremath{\mathrm{MHz}}\xspace}

\newcommand{\cm}{\ensuremath{\mathrm{cm}}\xspace}
\newcommand{\m}{\ensuremath{\mathrm{m}}\xspace}
\newcommand{\mm}{\ensuremath{\mathrm{mm}}\xspace}
\newcommand{\microm}{\ensuremath{\mu\mathrm{m}}\xspace}
\newcommand{\s}{\ensuremath{\mathrm{s}}\xspace}
\newcommand{\musec}{\ensuremath{\mu\mathrm{s}}\xspace}


%---------------------------------------------------------------------------------------------------
% include figures
%---------------------------------------------------------------------------------------------------
\newcommand{\myfig}[5]{
%% example:
% \myfig{}%% filename in figures folder
%       {width=0.5\textwidth,height=0.5\textheight}%% maximum width/height, aspect ratio will be kept
%       {}%% caption
%       {}%% optional (short) caption for list of figures
%       {}%% label
\begin{figure}%[htp]
  \begin{center}
     \includegraphics[keepaspectratio,#2]{figures/#1}
     \caption[#4]{#3}
     \label{#5} %% NOTE: always label *after* caption!
  \end{center}
  
\end{figure}
}




\documentclass{article}


%%=====================================================================================
%% drawing tikz
%%=====================================================================================
%
\usepackage{tikz}
\usetikzlibrary{positioning,shapes,arrows}%
\tikzstyle{memblock} = [draw, fill=blue!20, rectangle, 
    minimum height=6em, minimum width=3em]%
\definecolor{mygray}{cmyk}{0,0,0,0.4}%
\definecolor{mydarkgray}{cmyk}{0,0,0,0.7}%
\definecolor{mylightgray}{cmyk}{0,0,0,0.1}%

%________________________________________________________________
%tikz flow chart
\tikzstyle{decision} = [diamond, draw, fill=blue!20, 
    text width=4.25em, text badly centered, node distance=2cm, inner sep=0pt]
\tikzstyle{block} = [rectangle, draw, fill=blue!20, 
     text centered, rounded corners, minimum height=1.5em] 
     
\tikzstyle{block2} = [rectangle, draw, fill=orange!20, 
     text centered, rounded corners, minimum height=1.5em] 
     
\tikzstyle{block3} = [rectangle, draw, fill=green!20, 
     text centered, rounded corners, minimum height=1.5em] 
     
\tikzstyle{rect} = [rectangle, draw, fill=blue!20, text centered, minimum
height=1.5em, minimum width=5em]
    
    
    
\tikzstyle{line} = [draw, -latex']
\tikzstyle{cloud} = [draw, ellipse,fill=red!20, node distance=2cm,
    minimum height=1em]
    
\tikzstyle{txtblk} = [above, text centered]

% Define the layers to draw the diagram
\pgfdeclarelayer{background}
\pgfdeclarelayer{foreground}
\pgfsetlayers{background,main,foreground}

\usepackage[margin=0.5cm]{geometry}

\begin{document}


\begin{tikzpicture}[node distance=3cm]

\node (hash1) [txtblk, text width=7em] {\scriptsize{}};

\path (hash1.south)+(0,-0.25) node (config) [txtblk, text width=7em]
{\scriptsize{}};

\path (hash1.east)+(1.25,-0.2) node (hash2) [txtblk, text width=3em]
{\scriptsize{}};
\path (config.east)+(1.25,-0.2) node (id1) [txtblk, text
width=3em] {\scriptsize{}};

\path (hash2.east)+(1,-0.2) node (hash3) [txtblk, text width=3em]
{\scriptsize{}};
\path (id1.east)+(1,-0.2) node (update1) [txtblk, text
width=3.5em] {\scriptsize{}};

\path (hash3.east)+(1,-0.2) node (hash4) [txtblk, text width=3em]
{\scriptsize{}};
\path (update1.east)+(1,-0.2) node (update2) [txtblk, text
width=3.5em] {\scriptsize{}};

\path (update2.east)+(0.75,0) node (tpoint) [txtblk, text width=3em]
{\scriptsize{}};

\path (tpoint.east)+(0.5,-0.075) node (hashn) [txtblk, text width=3em]
{\scriptsize{}};
\path (hashn.south)+(0,-0.25) node (updaten) [txtblk, text
width=3.5em] {\scriptsize{}};

\path (hashn.east)+(1,-0.2) node (hashn1) [txtblk, text width=5em]
{\scriptsize{Hash ID x}};
\path (hashn1.south)+(0,-0.35) node (finish) [txtblk, text
width=3em] {\scriptsize{finish}};

\path (finish.east)+(0.75,0) node (digest) [txtblk, text
width=7em] {\scriptsize{Digest}};

\path (hash4.north)+(0,1) node (app) [txtblk, text
width=3em] {Application};

% This allow to create the rectangle
\begin{pgfonlayer}{background}
	  
  	%create  the lines of the rectangle  with an offset (x,y)         
	\path (config.west |- app.north)+(-0.25,0.25) node (a) {};
  	\path (finish.south -|  digest.east)+(0,-0.25) node (b) {};      
          
    % Combine the twos nodes above for creating the rectangle      
    \path[fill=blue!20,rounded corners, draw=black!50, dashed]
    (a) rectangle (b);  
            
\end{pgfonlayer}

%%%%%%%%%%%%%%%%%%%
% Interface 	  %
%%%%%%%%%%%%%%%%%%%


\path (config.south)+(0,-3) node (hash11) [txtblk, text width=7em]
{\scriptsize{}};

\path (hash11.south)+(0,-0.25) node (config2) [txtblk, text width=7em]
{\scriptsize{}};

\path (hash11.east)+(1.25,-0.2) node (hash21) [txtblk, text width=3em]
{\scriptsize{}};
\path (config2.east)+(1.25,-0.2) node (id2) [txtblk, text
width=3em] {\scriptsize{}};

\path (hash21.east)+(1,-0.2) node (hash31) [txtblk, text width=3em]
{\scriptsize{}};
\path (id2.east)+(1,-0.2) node (update11) [txtblk, text
width=3.5em] {\scriptsize{}};

\path (hash31.east)+(1,-0.2) node (hash41) [txtblk, text width=3em]
{\scriptsize{}};
\path (update11.east)+(1,-0.2) node (update21) [txtblk, text
width=3.5em] {\scriptsize{}};

\path (update21.east)+(0.75,0) node (tpoint2) [txtblk, text width=3em]
{\scriptsize{}};

\path (tpoint2.east)+(0.5,-0.075) node (hashn11) [txtblk, text width=3em]
{\scriptsize{}};
\path (hashn11.south)+(0,-0.25) node (updaten1) [txtblk, text
width=3.5em] {\scriptsize{}};

\path (hashn11.east)+(1,-0.2) node (hashn12) [txtblk, text width=3em]
{\scriptsize{}};
\path (hashn12.south)+(0,-0.25) node (finish2) [txtblk, text
width=3em] {\scriptsize{}};

\path (finish2.east)+(0.75,0) node (digest2) [txtblk, text
width=7em] {\scriptsize{Digest}};

\path (update21.south)+(0.25,-1) node (int) [txtblk, text
width=3em] {Interface};

% This allow to create the rectangle
\begin{pgfonlayer}{background}
	  
  	%create  the lines of the rectangle  with an offset (x,y)         
	\path (config2.west |- hash21.north)+(-0.25,0.25) node (c) {};
  	\path (int.south -|  digest2.east)+(0,-0.25) node (d) {};      
          
    % Combine the twos nodes above for creating the rectangle      
    \path[fill=blue!20,rounded corners, draw=black!50, dashed]
    (c) rectangle (d);  
            
\end{pgfonlayer}

\def\aboveint{(-0.9,-2.65)}


\path [draw, ->] (finish.south)+(0,-0.2) -- (10.3,-2.8);
\path [draw, <-] (digest.south)+(0,-0.45) -- (11.7,-2.8);


\end{tikzpicture}

\end{document}