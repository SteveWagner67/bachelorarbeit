%_____________________________________________________________________________________
%
%       Filename:  declaration.tex
%
%    Description:  Thesis Template HS Offenburg
%
%        Version:  1.0
%        Created:  13.11.2015
%       Revision:  none
%
%         Author:  B.Eng. Oliver Kehret, okehret@stud.hs-offenburg.de
%   Organization:  HS Offenburg, Offenburg, Germany
%      Copyright:  Copyright (c) 2015, B.Eng. Oliver Kehret
%
%          Notes:  Inspired by Andreas Walz, Tobias Neff and Karl Voith
%                
%_____________________________________________________________________________________
\newcommand{\mylanguage}{ngerman,american}
\newcommand{\mydispositioncolor}{0,0,0}
%% e.g., "30,103,182" (blue/turquois), "0,0,0" (black), ...
%% Color of the headings and so forth in RGB (red,green,blue) values.
\newcommand{\mycolorlinks}{false}  %% "true" or "false"
%% Enables or disables colored links (hyperref package).
\newcommand{\mytodonotesoptions}{}
%% e.g., "" (empty), "disable", ...
%% Options for the todonotes-package. If "disable", all todonotes will
%% be hidden (including todos).
%% ========================================================================
%%  Document metadata
%% ========================================================================
%% general metadata:
\newcommand{\myauthor}{Steve Wagner}  %% also used for PDF metadata (hyperref)
\newcommand{\mytitle}{}  %% also used for PDF metadata (hyperref)
\newcommand{\mysubject}{Extension and Integration of a Cryptographic Interface} 
%% also used for PDF metadata (hyperref)
\newcommand{\mykeywords}{creation, implementation, embetterTLS, LibTomCrypt}  %%
% also used for PDF metadata (hyperref)
%% this information is used only for generating the title page:
\newcommand{\myworktitle}{}  %% official type of work like ``Master theses''
\newcommand{\mygrade}{} %% title you are getting with this work like ``Master of ...''
\newcommand{\mystudy}{} %% your study like ``Arts''
\newcommand{\myuniversity}{Offenburg University of Applied Sciences} %% your university/school
\newcommand{\myinstitute}{Institut for reliable Embeeded Systems and Communication Electronics (ivESK)} %% affiliation
\newcommand{\myinstitutehead}{Prof. Sikora} %% head of institute 
\newcommand{\mysupervisor}{Dipl.-Phys. Andreas Walz} %% your supervisor
\newcommand{\myevaluator}{myprof} %% your evaluator
\newcommand{\myhomestreet}{Badstraße 24} %% your home street (with house number)
\newcommand{\myhometown}{Offenburg} %% your home town
\newcommand{\myhomepostalnumber}{77652} %% your postal number of home town
\newcommand{\mysubmissionmonth}{Septembre} %% month you are handing in
\newcommand{\mysubmissionyear}{2015} %% year you are handing in
\newcommand{\mysubmissiontown}{\myhometown} %% town of handing in (or \myhometown)
%% additional information for generic_documentation title page
%---------------------------------------------------------------------------------------------------
% names
%---------------------------------------------------------------------------------------------------
%___________________________________________________________________________________________________
% thing you should definitely use as macro (ease of translation etc.)
%___________________________________________________________________________________________________
\newcommand{\myref}[1]{\ref{#1}}
\newcommand{\eg}{e.g.\xspace} 
\newcommand{\vv}{vice versa\xspace} 
%---------------------------------------------------------------------------------------------------
% names the \xspace is very handy its intelligent enough to make no space before .
%---------------------------------------------------------------------------------------------------
\newcommand{\app}{Application\xspace}
\newcommand{\prov}{Provider\xspace}
\newcommand{\gci}{Generic Cryptographic Interface (GCI)\xspace}
\newcommand{\vaultic}{VaultIC\num{460}\xspace}
\newcommand{\atmel}{Atmel\xspace}
\newcommand{\atec}{ATECC\num{508}A\xspace}
\newcommand{\embeederssl}{\embtls}
\newcommand{\embtls}{emb::TLS\xspace}
\newcommand{\nist}{NIST\xspace}
\newcommand{\tls}{TLS\xspace}
\newcommand{\spi}{SPI\xspace}
\newcommand{\rsa}{RSA\xspace}
\newcommand{\dsa}{DSA\xspace}
\newcommand{\ecdsa}{ECDSA\xspace}
\newcommand{\tomcrypt}{LibTomCrypt\xspace}
\newcommand{\Table}{Table}
\newcommand{\Tables}{Tables}
\newcommand{\Figure}{Figure}
\newcommand{\Figures}{Figures}
\newcommand{\Subfigure}{Subfigure}
\newcommand{\Section}{Section}
\newcommand{\Sections}{Sections}
\newcommand{\Chapter}{Chapter}
\newcommand{\Chapters}{Chapters}
\newcommand{\Equation}{Equation}
\newcommand{\Equations}{Equations}
\newcommand{\Appendix}{Appendix}
\newcommand{\Appendices}{Appendices}
\newcommand{\Ref}{Ref.}
\newcommand{\Refs}{Refs.}


\documentclass{article}


%%=====================================================================================
%% drawing tikz
%%=====================================================================================
%
\usepackage{tikz}
\usetikzlibrary{positioning,shapes,arrows}%
\tikzstyle{memblock} = [draw, fill=blue!20, rectangle, 
    minimum height=6em, minimum width=3em]%
\definecolor{mygray}{cmyk}{0,0,0,0.4}%
\definecolor{mydarkgray}{cmyk}{0,0,0,0.7}%
\definecolor{mylightgray}{cmyk}{0,0,0,0.1}%

\usepackage[margin=0.5cm]{geometry}

%________________________________________________________________
%tikz flow chart
\tikzstyle{decision} = [diamond, draw, fill=blue!20, 
    text width=4.25em, text badly centered, node distance=2cm, inner sep=0pt]
\tikzstyle{block} = [rectangle, draw, fill=blue!20, 
     text centered, rounded corners, minimum height=1.5em] 
     
\tikzstyle{block2} = [rectangle, draw, fill=orange!20, 
     text centered, rounded corners, minimum height=1.5em] 
     
\tikzstyle{block3} = [rectangle, draw, fill=green!20, 
     text centered, rounded corners, minimum height=1.5em] 
     
\tikzstyle{rect} = [rectangle, draw, fill=blue!20, text centered, minimum
height=1.5em, minimum width=5em]
    
    
    
\tikzstyle{line} = [draw, -latex']
\tikzstyle{cloud} = [draw, ellipse,fill=red!20, node distance=2cm,
    minimum height=1em]
    
\tikzstyle{txtblk} = [above, text centered]

% Define the layers to draw the diagram
\pgfdeclarelayer{background}
\pgfdeclarelayer{foreground}
\pgfsetlayers{background,main,foreground}

\begin{document}


\begin{tikzpicture}[node distance=2.25cm]



%%%%%%%%%%%%%%%%%%%%%%%%%%%%%%%%%%%%%%%%%%%%%%%%%%
%	Application									 %
%%%%%%%%%%%%%%%%%%%%%%%%%%%%%%%%%%%%%%%%%%%%%%%%%%

% Create a node
\node (embtls) [block, text width=5.5em]{\embtls};

\path (embtls.east)+(-3.25,0) node (other1) [block, text width=2.5em] {\ldots};
\path (embtls.west)+(3.25,0) node (other2) [block, text width=2.5em] {\ldots};

% Create a text which in coordinate (-2,-0.5) of the middle of the south of
% embtls node
\path (embtls.south) +(-4,-0.4) node (app) [text width=0.5] {Applications};

% This allow to create the rectangle
\begin{pgfonlayer}{background}
	  
  	%create  the lines of the rectangle  with an offset (x,y)         
	\path (other1.west |- other1.north)+(-1.75,0.65) node (a) {};
  	\path (other2.south -|  other2.east)+(0.5,-0.65) node (b) {};      
          
    % Combine the twos nodes above for creating the rectangle      
    \path[fill=mylightgray!20,rounded corners, draw=black!50, dashed]
    (a) rectangle (b);  
             
    \path (embtls.north west)+(-0.2,0.2) node (blank) {};
            
\end{pgfonlayer}

%%%%%%%%%%%%%%%%%%%%%%%%%%%%%%%%%%%%%%%%%%%%%%%%%%
%	Interface									 %
%%%%%%%%%%%%%%%%%%%%%%%%%%%%%%%%%%%%%%%%%%%%%%%%%%

\node (cryptw) [block, below of=embtls, text width=5.5em, fill=green!60]{GCI};

\path (cryptw.south) +(-4,-0.4) node (int1) [text width=0.5] {Interface};

\begin{pgfonlayer}{background}
          
   	\path (cryptw.west |- cryptw.north)+(-3.25,0.65) node (c) {};
   	\path (cryptw.south -|  cryptw.east)+(2,-0.65) node (d) {};
          
	\path[fill=mylightgray!20,rounded corners, draw=black!50, dashed]
    (c) rectangle (d);           
            
\end{pgfonlayer}


%%%%%%%%%%%%%%%%%%%%%%%%%%%%%%%%%%%%%%%%%%%%%%%%%%
%	Crypto provider								 %
%%%%%%%%%%%%%%%%%%%%%%%%%%%%%%%%%%%%%%%%%%%%%%%%%%
\node (tomcr) [block, below of=cryptw]{\tomcrypt};
\path (tomcr.west)+(-1.5,0) node (vltc) [block, text width=5.5em] {\vaultic};
\path (tomcr.east)+(0.9,0) node (other) [block, text width=2.5em] {\ldots};

\path (tomcr.south) +(-2.75,-0.4) node (lib) {Crypto providers};

\begin{pgfonlayer}{background}
          
   \path (vltc.west |- tomcr.north)+(-0.5,0.65) node (e) {};
   \path (vltc.south -|  other.east)+(0.5,-0.65) node (f) {};  
                   
   \path[fill=mylightgray!20,rounded corners, draw=black!50, dashed]
   (e) rectangle (f);
            
\end{pgfonlayer}


\path [draw, ->] (tomcr.north) -- (cryptw.south);
\path [draw, ->] (vltc.north) -- (cryptw.210);
\path [draw, ->] (other.north) -- (cryptw.330);

\path [draw, ->] (cryptw.north) -- (embtls.south);
\path [draw, ->] (cryptw.north) -- (other1.south);
\path [draw, ->] (cryptw.north) -- (other2.south);


\end{tikzpicture}

\end{document}