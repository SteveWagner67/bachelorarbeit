%_____________________________________________________________________________________
%
%       Filename:  preamble.tex
%
%    Description:  Thesis Template HS Offenburg
%
%        Version:  1.0
%        Created:  13.11.2015
%       Revision:  none
%
%         Author:  B.Eng. Oliver Kehret, okehret@stud.hs-offenburg.de
%   Organization:  HS Offenburg, Offenburg, Germany
%      Copyright:  Copyright (c) 2015, B.Eng. Oliver Kehret
%
%          Notes:  Inspired by Andreas Walz, Tobias Neff and Karl Voith
%                
%_____________________________________________________________________________________
\documentclass{beamer}
\usetheme{hsogei}
%%=====================================================================================
%% General presentation
%%=====================================================================================
% language adaptions, change in main.tex (english, ngerman, american)
\usepackage[\mylanguage]{babel}									
% nicer quotes  
\usepackage[%
            autostyle,          % adapts language setting
            strict,             % turns warnings into errors 
            english=american    % use american quotes style
]{csquotes}

%%=====================================================================================
%% Selection of useful packages
%%=====================================================================================
%%
%%=====================================================================================
%% SIUNITSX -- simplified usage of SI-units
%%=====================================================================================
\usepackage[% 
            exponent-product = \cdot, % use \cdot instead * for exponent product
            binary-units=true,
            load-configurations=binary, 
            load-configurations=abbreviations, 
]{siunitx}	              
%%=====================================================================================
%% ifthen and todonotes puts to-do-notes in the printed document if you want 
%%=====================================================================================
%% used to disable todonotes package
\usepackage{ifthen}                                         
%% pre-define ifthen-boolean variables:
\newboolean{myaddcolophon}
\newboolean{myaddlistoftodos}
%
% currently american is not supported by todonotes but english is fine as it's only
% effects todonotes and missingfigures
\usepackage[\mytodonotesoptions,english]{todonotes}
%%=====================================================================================
%% Sourcecode printing
%%=====================================================================================
\usepackage{listings}				% include source code
									% ftp://ftp.tex.ac.uk/tex-archive/macros/latex/contrib/listings/listings.pdf
\lstset{% 							% options for representation of source code
  backgroundcolor=\color{light-gray},   % choose the background color; you must add \usepackage{color} or \usepackage{xcolor}
  basicstyle=\footnotesize,        % the size of the fonts that are used for the code
  breakatwhitespace=false,         % sets if automatic breaks should only happen at whitespace
  breaklines=true,                 % sets automatic line breaking
  captionpos=b,                    % ses the caption-position to bottom
  commentstyle=\color{dark-green}, % comment style
  deletekeywords={...},            % if you want to delete keywords from the given language
  escapeinside={\%*}{*)},          % if you want to add LaTeX within your code
  extendedchars=false,              % lets you use non-ASCII characters; for 8-bits encodings only, does not work with UTF-8
  frame=lines,                     % adds a frame around the code
  keepspaces=true,                 % keeps spaces in text, useful for keeping indentation of code (possibly needs columns=flexible)
  keywordstyle=\color{blue},       % keyword style
  language=C,                      % the language of the code
  morekeywords={*,...},            % if you want to add more keywords to the set
  numbers=left,                    % where to put the line-numbers; possible values are (none, left, right)
  numbersep=8pt,                   % how far the line-numbers are from the code
  numberstyle=\tiny\color{gray},   % the style that is used for the line-numbers
  rulecolor=\color{black},         % if not set, the frame-color may be changed on line-breaks within not-black text (e.g. comments (green here))
  showspaces=false,                % show spaces everywhere adding particular underscores; it overrides 'showstringspaces'
  showstringspaces=false,          % underline spaces within strings only
  showtabs=false,                  % show tabs within strings adding particular underscores
  stepnumber=1,                    % the step between two line-numbers. If it's 1, each line will be numbered
  stringstyle=\color{blue},        % string literal style
  tabsize=2,                       % sets default tabsize to 2 spaces
  title=\lstname,                  % show the filename of files included with \lstinputlisting; also try caption instead of title
  %numberbychapter=false
}
%%=====================================================================================
%% Math 
%%=====================================================================================
\usepackage{amssymb,amstext} %% predefiened symbols e.g. \nparallel
\usepackage[%
            fleqn,%equations aligned in a fixed distance from the left
            tbtags, %where the equation number is placed here bottom or top
]{mathtools} %% loads amsmath package

%%=====================================================================================
%% Tables, figures etc.
%%=====================================================================================
%
%% nice rule's for tables try \toprule \midrule \bottomrule  
\usepackage{booktabs}
%% set width of table and more
\usepackage{tabularx}										% creates tables
%
%% rotate tables and figures
\usepackage{rotating}
%
%
%% define caption style
\usepackage[font=scriptsize, width=0.9\textwidth, format=plain, labelfont=bf]{caption}
%%
\usepackage{subfigure}

%%=====================================================================================
%% some utility stuff
%%=====================================================================================
%
\usepackage[]{acronym}				% for usage of abbreviations
%
% improved typographical settings
\usepackage[%
    protrusion=true, %
    factor=900       %
]{microtype}
%
%% switch of extra space after punctuation
\frenchspacing 
%
%% switches to Palatino with small caps and old style figures
\usepackage[%
sc,%
osf,%
]{mathpazo}
%
%% customize item look
%\usepackage{enumitem}
%% kills space between items
%\setlist{noitemsep}
%doc% For additional special characters available by \verb#\ding{}#
\usepackage{pifont}  %% Sonderzeichen fuer Titelseite \ding{}
%
%doc% This package is required for intelligent spacing after commands
\usepackage{xspace}
%
%
%doc% This package offers strikethrough command \verb+\sout{foobar}+.
\usepackage[normalem]{ulem}
%
%
%doc% Create framed, shaded, or differently highlighted regions that can 
%doc% break across pages.  The environments defined are 
%doc% \begin{itemize}
%doc%   \item framed: ordinary frame box (\verb+\fbox+) with edge at margin
%doc%   \item shaded: shaded background (\verb+\colorbox+) bleeding into margin
%doc%   \item snugshade: similar
%doc%   \item leftbar: thick vertical line in left margin
%doc% \end{itemize}
\usepackage{framed}
%
%doc% For example on title pages you might want to have a logo on the upper right corner of
%doc% the first page (only). The package \texttt{eso-pic} is able to place things on absolute
%doc% and relative positions on the whole page.
\usepackage{eso-pic} %%
%% for what ????
%\usepackage{lastpage}										% get total number of pages
%

%%=====================================================================================
%% drawing tikz
%%=====================================================================================
%
\usepackage{tikz}
\usetikzlibrary{positioning,shapes,arrows}%
\tikzstyle{memblock} = [draw, fill=blue!20, rectangle, 
    minimum height=6em, minimum width=3em]%
\definecolor{mygray}{cmyk}{0,0,0,0.4}%
\definecolor{mydarkgray}{cmyk}{0,0,0,0.7}%
\definecolor{mylightgray}{cmyk}{0,0,0,0.1}%

%________________________________________________________________
%tikz flow chart
\tikzstyle{decision} = [diamond, draw, fill=blue!20, 
    text width=4.25em, text badly centered, node distance=2cm, inner sep=0pt]
\tikzstyle{block} = [rectangle, draw, fill=blue!20, 
     text centered, rounded corners, minimum height=1.5em] 
     
\tikzstyle{block2} = [rectangle, draw, fill=orange!20, 
     text centered, rounded corners, minimum height=1.5em] 
     
\tikzstyle{block3} = [rectangle, draw, fill=green!20, 
     text centered, rounded corners, minimum height=1.5em] 
     
\tikzstyle{rect} = [rectangle, draw, fill=blue!20, text centered, minimum
height=1.5em, minimum width=5em]
    
    
    
\tikzstyle{line} = [draw, -latex']
\tikzstyle{cloud} = [draw, ellipse,fill=red!20, node distance=2cm,
    minimum height=1em]
    
\tikzstyle{txtblk} = [above, text centered]

% Define the layers to draw the diagram
\pgfdeclarelayer{background}
\pgfdeclarelayer{foreground}
\pgfsetlayers{background,main,foreground}

