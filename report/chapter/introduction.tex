\chapter{Introduction}

\section{Cryptography} % around 3/4 pages

Cryptography uses mathematical techniques for the security of datas transmitted
over an insecure network. \newline
Cryptanalysis is the complementary of the cryptography
with the focus on the defeat of the cryptographic mathematical
techniques.\newline

The security of the information is definied into:
\begin{itemize}
  \item Confidentiality or privacy\newline
  No one, except whom is intended, can understand the transmitted datas
  \item Integrity\newline
  No one can alter the transmitted message without the alteration is being
  detected
  \item Authentication \newline
  The sender and the receiver can identify the destination of the datas and
  identify themself
  \item Non-repudiation\newline
  The sender cannot deny at a later stage the tranmission of the datas\newline
\end{itemize}

The mathematical cryptographic techniques are grouped into several algorithms:
\begin{itemize}
  \item Hash algorithm
  \item Signature algorithm
  \item Symmetric cipher algorithm
  \item Asymmetric cipher algorithm
\end{itemize}

\newpage

\subsection{Hash algorithm}

\begin{figure}[!ht]
\centering
%\frame{
% trim: left, bottom, right, up
\includegraphics[trim=3.5cm 22.5cm 7.cm 0cm]{figures/hash.pdf}
\caption{Scheme of an hash operation\newline}
\label{fig:hash}
%}

\end{figure}

A cryptographic algorithm is considered pratically impossible to invert, meaning
that impossible to recreate the input data (message) with the digest (output of
the hash).

Tha main properties of a hash function are:
\begin{itemize}
  \item it's quick to compute the digest for any message
  \item it's infeasible to generate a message from its digest
  \item it is infeasible to modify a message without changing the digest
  \item it is infeasible to find two different messages with the same
  digest.\newline
\end{itemize}

On Figure \ref{fig:hash}, the message is hashed and the result (digest) is
sent to the other peer with the original message.\newline Then the other peer
hashs the message too (with the same algorithm) and compares the two digests to
know if the message has been changed during the transmission.\newline 
This is the princip of integrity.

\newpage

\subsection{Signature algorithm}
\label{intro_sign}

\begin{figure}[!ht]
\centering
%\frame{
% trim: left, bottom, right, up
\includegraphics[trim=2cm 22.25cm 7.5cm 0cm]{figures/signature.pdf}
\caption{Scheme of a signature operation\newline}
\label{fig:sign}
%}

\end{figure}

A digital signature is a mathematical princip to demonstrate the authenticity of
a message.\newline
It infeasible to generate the original message with the signature.

A valid digital signature allows to be sure that the incoming message comes from
the peer we are communicated with and not from someone else
(authenticity).\newline
With digital signature the sender cannot deny having sended the message
(non-repudiation).\newline
It allows too to be sure that the message has not been corrupted during the
transmission (integrity).\newline

The princip of a digital signature is shown figure \ref{fig:sign}.\newline
The message is signed with the private key (no one has this key too) and the
result (signature) is sent to the other peer (peer B).\newline
The public key of the peer A has already be sent previously. With this key, peer
B can verify the signature (only with the public key of peer A) and be sure that
the message sends with it comes from peer A and not from someone else.


\newpage

\subsection{Symmetric cipher algorithm}
\label{intro_sym_cipher}

\begin{figure}[!ht]
\centering
%\frame{
% trim: left, bottom, right, up
\includegraphics[trim=1cm 23.25cm 4cm 0cm]{figures/sym_cipher.pdf}
\caption{Scheme of a symmetric cipher operation}
\label{fig:sym}
%}

\end{figure}

Symmetric-key algorithms are algorithms for cryptography that use the same
cryptographic keys for both encryption of plaintext and decryption of
ciphertext in a communication.\newline
The key is often named shared secret key.\newline

There is two kind of symmetric encryption:
\begin{enumerate}
  \item Stream ciphers, which the encryption is done only for one digit
  (typically bytes) of a message at a time.
  \item Block ciphers, which take a number of bits and encrypt them as a single
  unit, padding the plaintext so that it is a multiple of the block
  size.\newline
\end{enumerate}

Figure \ref{fig:sym} represents the process for encryption and decryption with a
symmetric key.

The shared key has already been exchanged between the two peers.\newline
Peer A encrypts the message (plaintext) with this key and sends the result (the
encrypted data or ciphertext) to peer B.\newline
Than peer B uses the same key but to decrypt the encrypted data (ciphertext) and
then reads the plaintext sent by the peer A.\newline


No one who doesn't have this key can understand this message over the insecure
network represents in red figure \ref{fig:sym}.



\newpage

\subsection{Asymmetric cipher algorithm}
\label{intro_asym_cipher}


\begin{figure}[!ht]
\centering
%\frame{
% trim: left, bottom, right, up
\includegraphics[trim=1cm 23.25cm 4cm 0cm]{figures/asym_cipher.pdf}
\caption{Scheme of an asymmetric cipher operation}
\label{fig:asym}
%}

\end{figure}

Asymmetric algorithm is a method for encryption and decryption of messages with
public and private keys. \newline
One peer generates the private and public key together, keep the private key
(meaning that he is the only one to have it) and sends the public key to every
one he wants to communicate with.\newline
With the public key everyone can encrypt a message, which no one can understand
through the insecure network and no one can decrypt it, except this one who has
the private key.\newline

Figure \ref{fig:asym} shows the princip of encryption and decryption with
asymmetric algorithms.\newline
Peer B is this one who has generated the public and private keys and has already
sent the public key to peer A.\newline
Peer A encrypts the message with the public key of peer B. This
message is than sends to peer B over the insecure network.\newline
Thanks to the private key can peer B decrypts the message of peer A.\newline

Thanks to the princip of private and public keys, asymmetric algorithm allows
privacy (like symmetric algorithm) but integrity of data too, because only this
one who creates the keys has the private key for decryption.


\newpage

\subsection{Diffie-Hellman}

\begin{figure}[!ht]
\centering
%\frame{
% trim: left, bottom, right, up
\includegraphics[trim=0cm 20cm 8cm 0cm]{figures/diffie_hellman.pdf}
\caption{Scheme of a Diffie-Hellman operation}
\label{fig:dh}
%}

\end{figure}

Diffie-Hellman Key Exchange etablishes a shared secret key between two peers
that can be used for secret communication for exchanging data over an insecure
network.\newline
The princip of Diffie-Hellman Key Exchange is to begin with asymmetric keys and
to finish with symmetric key. \newline
It makes sure that the both parties
participate in the generation of the symmetric key.\newline

Figure \ref{fig:dh} shows the princip of the Diffie-Hellman key
exchanges.\newline 
Peer A creates the domain parameters and a private key only (not the
public).\newline With the domain parameters and the private key can peer A
calculates his public key.\newline
He sends than the domain parameters and his public key to peer B.\newline
Peer B generates a private key too and calculates his public key with the domain
parameters from peer A and its own private key.\newline
Peer B sends than his public key to peer A.\newline
To finish, each one calculates the shared secret (which is the same for the
twice) with the public key of the other peer and it's own private key.\newline

The shared secret is the same for the two peers and can be used for encryption
and decryption.

\newpage


\section{SSL/TLS protocol} %around 1/2 pages 
\label{intro_tls}

\begin{figure}[!ht]
\centering
%\frame{
% trim: left, bottom, right, up
\includegraphics[trim=0cm 20cm 8cm 0cm]{figures/tls_osi.pdf}
\caption{Placement of TLS in OSI model}
\label{fig:osi}
%}

\end{figure}

Transport Layer Security (TLS) is a client/server protocol that provides
different basic security services for the communication between peers:
\begin{itemize}
  \item Authentication (both peer and data origin authentication)
  services
  \item Connection confidentiality services
  \item Connection integrity services (without recovery)
\end{itemize}

This security layer is situated between the transport and the application layer
on the OSI model (see figure \ref{fig:osi})

This security protocol is often uses in:
\begin{itemize}
  \item E-commerce website for secured transaction and client authentication
  acces
  \item Remote access
  \item Web browsers to browse the Internet
  \item Simple Mail Transfer Protocol (SMTP)
  \item Virtual Private Network (VPN)
\end{itemize}

