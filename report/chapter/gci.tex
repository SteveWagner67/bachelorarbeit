\chapter{Generic Cryptographic Interface (GCI)}
\label{gci}

The goal of the interface is to have a base of the cryptographic algorithms,
which are described in the introduction, chapter (\ref{intro}).
The interface can also be used by applications which need cryptography and can
use different providers, i.e open-source software libraries as hardware-based
cryptographic modules.
New algorithms shall be easy to add in the interface, if new algorithms have to
be used and are still not implemented in the interface.
With the interface several different configurations shall be used, meaning that
no implicit parameters have to be written in functions of the interface.
This configuration shall then have the possibility to be easily used to compute
results with data added at any time.

The hardware-based cryptographic modules use a key management, which is not
used.
This institute wants to use this key management, that's why the interface shall
have a key management too.

After the cryptographic algorithms have been understood, the design of the
interface could be done.
As shown in the chapter introduction (\ref{intro}), the main cryptographic
algorithms are:

\begin{itemize}[noitemsep]
  \item Hash, which is oft used in signature and MAC algorithms
  \item Signature, which is widely used and also very important
  \item Message Authentication Code (MAC), which has some same properties as
  signatures, but some difference too, which make it important to have it in the
  interface.
  \item Cipher (symmetric and asymmetric), which are needed to encrypt and
  decrypt data
\end{itemize}


\section{Context}
\label{gci_ctx}
%1- what we need
% 2- Explain the solution of the context

Several different configurations are possible for a cryptographic algorithm.
These configurations should not have parameters, which are implicitly written in
the functions of the interface, but all of them has to be configurable from the
application part.
Furthermore, data can be added to the algorithm at any time, meaning that the
state of the algorithm, the configuration and the previously added data, have to
be saved somewhere. 
The result can be computed, only when the application needs it, that's why
should all the data and the configuration be saved somewhere too.

That's why the principle of the context is used. It represents the
state of the stateful algorithms.

Through the contexts, these informations, the configuration and the data, are
encapsulated and referred by an ID, which is used by the application when
other data has to be added or when the result has to be computed.

The contexts also allow to the application part to add data to an
algorithm by only passing the ID and this data at any time. 
The interface knows that this information has to be added to the context
with the referred ID.
The context keeps the configuration and the data till it's removed or a result
is computed.
When a result is computed, the context cannot be used again, because
the data are not saved in the interface, but added to a function from the
provider and this one removes the data when the result are computed.
The context should then be removed and created again.
This can be a problem (see section \ref{gci_hash} and \ref{gci_sign_mac}), but is
solved and explained in the section \ref{gci_cl_ctx}.

The cryptographic algorithms, which the principle of context is used, are:
\begin{itemize}[noitemsep]
  \item Hash algorithm
  \item Signature/MAC algorithm
  \item Cipher algorithm (symmetric and asymmetric)
  \item Diffie-Hellman\newline
\end{itemize}

\section{Cryptographic services}
\subsection{Hash}
\label{gci_hash}

A hash algorithm is an algorithm which is impossible to compute the input with
the generated output (digest).
It's used in some signature and Message Authentication Code (MAC), that's why is
essential to have it in the interface.
The hash algorithm in the interface is split into three main functions:
\begin{enumerate}[noitemsep]
  \item Configuration of the hash
  \item Update of data
  \item Calculation of the digest
  
\end{enumerate}

\subsection*{Configuration of the hash}

The hash algorithm has only one parameter to be configured, which is which
algorithm should be used for hashing.
The hash algorithm implemented in the interface are:
\begin{itemize}[noitemsep]
  \item MD5
  \item SHA1
  \item SHA224
  \item SHA256
  \item SHA384
  \item SHA512
\end{itemize}

\begin{figure}[h]
\centering
%\frame{
% trim: left, bottom, right, up
\includegraphics[trim=0cm 20cm 13cm 0cm]{figures/hash_example_config.pdf}
\caption{Configuration of a hash algorithm}
\label{fig:gci_hash_config}
%}
\end{figure}

When the configuration is done this one should be sent to the interface which
will save it in a context.
The interface will then return an ID of the context, which corresponds of
where is saved the configuration.
This is shown on figure \ref{fig:gci_hash_config} 

\subsection*{Update of data}

When the configuration is done, several updates can be done.
The principle on an update is to add a data which we want to hash.

\begin{figure}[!ht]
\centering
%\frame{
% trim: left, bottom, right, up
\includegraphics[trim=8cm 20cm 12cm 0cm]{figures/hash_example_update.pdf}
\caption{Update of a hash algorithm}
\label{fig:gci_hash_update}
%}
\end{figure}

As shown in the figure \ref{fig:gci_hash_update}, the ID received in the
configuration part has to be used to add a data.
The ID and the data has therefore sent to the interface, which knows through the
context ID that it should hash the data with the configuration saved in this
context.

\subsection*{Calculation of the digest}

When all the data we want to hash are sent to the interface, the interface can
calculate the digest.

\begin{figure}[!ht]
\centering
%\frame{
% trim: left, bottom, right, up
\includegraphics[trim=18.5cm 20.5cm 13.5cm 0cm]{figures/hash_example_finish.pdf}
\caption{Calculation of the digest}
\label{fig:gci_hash_finish}
%}
\end{figure}

As shown in figure \ref{fig:gci_hash_finish}, by passing the ID which contains
the configuration and all the updated data, the interface will, through the
provider, calculate the digest.
One of the disadvantages of this part is
when the digest is calculated, all the updated data and the configuration are lost, meaning that we cannot use them
again to calculate another hash with other data.
This problem is solved in chapter \ref{gci_cl_ctx}

\subsection{Generate key pair}
\label{gci_gen_key}

Some parts of cryptography need keys to work, like signatures and ciphers.
The key can be a symmetric, which is the same key for the two peers
in a communication, or asymmetric, which is different for the two peers in a
communication.
The goal of this function is to generate the asymmetric keys, public key and
private key. To generate a symmetric key, other methods should be used like
Diffie-Hellman (see section \ref{gci_dh}).
In the interface, three kinds of key pair can be generated, each having
differents configuration:
\begin{itemize}
  \item RSA key pair\newline
  The size of the key should be configured (1024 bits, 2048 bits, 4096 bits,
  etc.)
  \item DSA key pair\newline
  The domain parameters can be configured if needed or internally generated
  \item ECDSA key pair (Elliptic curve)\newline
  The type of the elliptic curve should be configured
\end{itemize}

When the configuration is done, this one should be sent to the
interface.
The keys will then be generated through a cryptographic provider.
The keys are, however, returned as IDs, which is one of the principles of the
interface. For more details about the key ID see chapter key management
\ref{gci_key_mng}

\subsection{Digital signature - Message Authentication Code (MAC)}
\label{gci_sign_mac}

The digital signature and the MAC are widely used today. Both provide integrity
and authentication of a message. Only the digital signature provides
non-repudiation more. MAC is much faster than digital signature through the use
of symmetric keys.
As some specifications of certain provider, the signature/MAC function has two
possibilities of use:
\begin{enumerate}[noitemsep]
  \item Signing
  \item Verifying\newline
\end{enumerate}

Each one is split into three functions:
\begin{itemize}[noitemsep]
  \item Configuration of the signature
  \item Update of data
  \item Generate a signature/Verify a signature\newline
\end{itemize}


\subsection*{Configuration of the signature}
For the generation and verification of a signature this part is the same (only
the name of the function changes).
First should the signature be configured.
Several parameters have to be configured which are:
\begin{itemize}
  \item the signature/MAC algorithm, which could be:
  \begin{itemize}[noitemsep]
    \item RSA
    \item DSA
    \item ECDSA
    \item CMAC (MAC from Block Ciphers)
    \item HMAC (MAC from Hash functions)
  \end{itemize}
  \item A hash algorithm, if in the updated data should be first hashed before
  signing, or if the HMAC algorithm is used
  \item The padding, if the RSA algorithm is used
  \item The block mode, the padding and the initialization vector, if the CMAC
  algorithm is used
  \item The private key, if RSA, DSA or ECDSA is used to generate a signature or
  the public key if the same algorithm is used to verify a signature
\end{itemize}

\begin{figure}[!ht]
\centering
%\frame{
% trim: left, bottom, right, up
\includegraphics[trim=0cm 20cm 13.5cm 0cm]{figures/sign_example_config.pdf}
\caption{Configuration of a signature/MAC algorithm}
\label{fig:gci_sign_config}
%}
\end{figure}

As shown figure \ref{fig:gci_sign_config}, when the configuration is done, this
one is saved in a context in the interface. An ID of the context is returned,
which indicated where is the configuration saved.

\subsection*{Update of data} 

When the configuration is done, several updates could be done.\newline
The principle of the update is to add data which will be used to generate or
verify a signature.\newline

\begin{figure}[!ht]
\centering
%\frame{
% trim: left, bottom, right, up
\includegraphics[trim=8cm 20cm 12cm 0cm]{figures/sign_example_update.pdf}
\caption{Update of a signature/MAC algorithm}
\label{fig:gci_sign_update}
%}
\end{figure}

As shown in figure \ref{fig:gci_sign_update}, to use the correct configuration,
the ID of the context, returned when the configuration is saved, should be
used.
The data and the ID is then sent to the interface, which will sign this data
with the configuration saved in this context.


\subsection*{Generate a signature/Verify a signature}
\begin{figure}[!ht]
\centering
%\frame{
% trim: left, bottom, right, up
\includegraphics[trim=18.5cm 20.5cm 13.5cm 0cm]{figures/sign_example_finish.pdf}
\caption{Calculation of the signature}
\label{fig:gci_sign_finish}
%}
\end{figure}
In this part the generation and the verification are different.
For the generation , the whole updated data and the configuration will be signed
with the private key added in the configuration for the digital
signature but with a symmetric key for the MAC.

For the verification, the signature we want to verify should be added to the
function.
Then the updated data will be signed, but with the public key for the
digital signature, and the same symmetric key as the generation of the
signature for the MAC.
Then the added signature (which is done with the
private key) could be compared with the ``signature'' computed to verify.
The most important part of the verification is that the private key, which
the signature is done and the public key for the verification, should be
generated together for the digital signature and the same symmetric key should
be used for the MAC.

\subsection{Cipher (symmetric and asymmetric)}
\label{gci_ciph}
A cipher, as explained in \ref{intro_cipher}, which
could be symmetric or asymmetric, is an algorithm for encrypting and decrypting
data.
This concept is therefore used in the interface and split into three main
functions which are:
\begin{itemize}[noitemsep]
  \item Configuration of the cipher
  \item Encryption of a plaintext
  \item Decryption of a ciphertext
\end{itemize}

\subsection*{Configuration of the cipher}
Several parameters should be configured for the cipher algorithm and depend
particularly of which cipher algorithm is used.
The cipher algorithm is split into three main algorithm:

\begin{itemize}
  \item Symmetric stream cipher algorithm\newline
Today very deprecated but is, however, implemented in the interface if
comparison has to be done.
Only the RC4 stream cipher is implemented in the interface.
Another stream ciphers can, however, be easily added in the interface if
needed.
Nothing more as the algorithm has to be configured for the use of
it.
  \item Symmetric block cipher algorithm\newline
  Three kinds of symmetric block cipher algorithms are used today and
therefore implemented in the interface:
\begin{itemize}[noitemsep]
  \item Data Encryption Standard (DES)
  \item 3DES, three subsequent DES encryption
  \item Advanced Encryption Standard (AES)  
\end{itemize}
Each symmetric block cipher algorithm needs a mode of operation, named block
mode in the interface, which depends on the size of data we want to
encrypt.\newline
The block modes implemented in the interface are:
\begin{itemize}[noitemsep]
  \item Electronic Code Book mode (ECB)
  \item Cipher Block Chaining mode (CBC)
  \item Cipher Feedback mode (CFB)
  \item Output Feedback mode (OFB)
  \item Counter mode (CTR)
  \item Galois Counter Mode (GCM)
\end{itemize}
  \item Asymmetric algorithm\newline
  Only the RSA algorithm is implemented for this part of the interface.
  In practice to use the RSA algorithm this one should use a padding to increase
  the security of it.
  The best known padding in the Public-Key Cryptography Standard (PKCS) and is,
of course, implemented in the interface.\newline  
\end{itemize} 
The most important thing for a cipher is, of course, the key!
For a symmetric cipher, stream or block, the key is only a shared key.
For an asymmetric cipher, if an encryption will be done, a public key should
be added. If a decryption will be done, a private key should be added to the
configuration.

\begin{figure}[!ht]
\centering
%\frame{
% trim: left, bottom, right, up
\includegraphics[trim=8.5cm 20cm 12.5cm 0cm]{figures/cipher_example_config.pdf}
\caption{Configuration of a cipher algorithm}
\label{fig:gci_cipher_config}
%}
\end{figure}

When the configuration is done this one should be sent to the interface
which will save it in a context.
The interface returned an ID of the context, which corresponds of where is saved
the configuration if this one should be used in the future.
The key added to the function is an ID of the key which is already saved in
the interface.
This principle is shown in figure \ref{fig:gci_cipher_config}

\subsection*{Encryption of a plaintext}
\begin{figure}[!ht]
\centering
%\frame{
% trim: left, bottom, right, up
\includegraphics[trim=13cm 20.5cm 10cm 0cm]{figures/cipher_encrypt.pdf}
\caption{Encryption of a plaintext}
\label{fig:gci_cipher_encrypt}
%}
\end{figure}
When the configuration is done, a encryption can be done.
To encrypt data, the ID of the context (where is the configuration saved)
should be added to the function with the data to encrypt (plaintext).
The interface will, through a provider, calculate the ciphertext of the
plaintext with the configuration saved previously in the context.
This principle is shown figure \ref{fig:gci_cipher_encrypt}


\subsection*{Decryption of a ciphertext}

\begin{figure}[!ht]
\centering
%\frame{
% trim: left, bottom, right, up
\includegraphics[trim=13cm 20.5cm 10cm 0cm]{figures/cipher_decrypt.pdf}
\caption{Decryption of a ciphertext}
\label{fig:gci_cipher_decrypt}
%}
\end{figure}
When the configuration is done, a decryption can be done.
To decrypt a data, the ID of the context (where is the configuration saved)
should be added to the function with the data to decrypt (ciphertext).
The interface will, through a provider, calculate the plaintext of the
ciphertext with the configuration saved previously in the context.
This principle is shown figure \ref{fig:gci_cipher_decrypt}


\subsection{Diffie-Hellman}
\label{gci_dh}

Diffie-Hellman key exchange is a specific method of securely exchanging
cryptographic keys over an insecure network. It starts with asymmetric keys to finish with a
symmetric, shared key which is the same for the two peer in a communication.

The Diffie-Hellman key exchange should therefore have the possibility to
generate key pairs and to calculate the shared key.

That's why in the interface the Diffie-Hellman protocol is split into three main
functions:
\begin{itemize}[noitemsep]
  \item Configuration of the Diffie-Hellman protocol
  \item Generation key pair
  \item Calculation of the shared key
\end{itemize}


\subsubsection*{Configuration of the Diffie-Hellman protocol}
The Diffie-Hellman key exchange and the Elliptic Curve of Diffie-Hellman key
exchange can be used in the interface.
For the configuration, one of these key exchanges should be chosen.

For the Diffie-Hellman key exchange the domain parameters can be added. If no
domain parameters have been added, the interface will, through the provider,
generate them.

For the Elliptic Curve of Diffie-Hellman key exchange, the type of curve should
be configured.

\begin{figure}[!ht]
\centering
%\frame{
% trim: left, bottom, right, up
\includegraphics[trim=0cm 20cm 13.5cm 0cm]{figures/gci_dh_config.pdf}
\caption{Configuration of a Diffie-Hellman protocol}
\label{fig:gci_dh_config}
%}
\end{figure}

As shown on figure \ref{fig:gci_dh_config}, when the configuration is done, this
one should be sent to the interface which will save it in a context.
An ID of the context will be returned to the application, which indicates the
placement of the context, also of the configuration.


\subsubsection*{Generation of the key pair}

When the configuration is done, key pair can be generated. It's the same
principle as explain in the chapter \ref{gci_gen_key}, but for the
Diffie-Hellman it's a little bit different. The private key of the Diffie must not go out of the
interface. That's why when the key should be generated only the ID of the public
key will go out of the function. The private key will be saved in the
context, same context where is the configuration saved. This is why
Diffie-Hellman is not implemented with the asymmetric cipher algorithm, where the public key and private key are going out of the function and don't stay in
the context. This is shown on figure \ref{fig:gci_dh_gen_key}

\begin{figure}[!ht]
\centering
%\frame{
% trim: left, bottom, right, up
\includegraphics[trim=8.5cm 20cm 13cm 0cm]{figures/gci_dh_gen_key.pdf}
\caption{Generation of Diffie-Hellman key pair}
\label{fig:gci_dh_gen_key}
%}
\end{figure}


\subsubsection*{Calculation of the secret key}

When the public key of the peer is received, the shared key could be generated.
To do it, the context with the configuration and the private, and the public key
of the peer should be added to the interface. This one will, through a provider
calculate the shared key and returned the ID of this one. This is shown on
figure \ref{fig:gci_dh_calc_key}

\begin{figure}[!ht]
\centering
%\frame{
% trim: left, bottom, right, up
\includegraphics[trim=8.5cm 20cm 13cm 0cm]{figures/gci_dh_calc_key.pdf}
\caption{Calculation of the Diffie-Hellman secret key}
\label{fig:gci_dh_calc_key}
%}
\end{figure}


\subsection{Random number generator}
\label{gci_rng}

A random number generator is a computational or physical devices for generating
a sequence of numbers which are impossible to predict better than a random
chance.\newline
It exists two main random number generator:
\begin{enumerate}
  \item True Random Number Generators (TRNG)\newline
  Are characterized that the output cannot be reproduced. They are based on
  physical processes like semiconductor noise or clock jitter in digital
  circuits, etc..
  \item Pseudorandom Number Generators (PNRG)
  Generated sequences which are computed from an initial seed value. PRNGs
  possess good static properties, meaning their output approximates a sequence
  of true random numbers. This is shown on figure \ref{fig:gci_prng}
\end{enumerate}

For the interface, Random number generators are very important, to generate keys
or for the ciphers, for example.

For the use of a True Random Number Generator (TRNG), with hardware-based
cryptographic modules for example, only the function to get a random number is
needed.

For the use of Pseudo Random Number Generator (PRNG), with a
cryptographic software library for example, a function to generate the initial
seed value is, furthermore, needed.


\section{Clone of context}
\label{gci_cl_ctx}
As explained in \ref{gci_hash} and \ref{gci_sign_mac} when the digest (for the hash)
and the signature (for the digital signature/MAC) is calculated, no more data
can be added to the context.
This is a problem for the use of this interface in TLS projects (\embtls for
example).
Several solutions were introduced which are:
\begin{enumerate}
  \item Use two contexts at the same time.\newline
  This wasn't very efficient, because we should know at the beginning the
  number of times a digest will be calculated, which determines the amount of
  context we have to create at the beginning.
  \item Create a context when the digest is calculated.\newline
  The disadvantage of this idea was that the whole data use previously has to be
  saved. For applications, which are used in embedded systems, like \embtls, For
  systems, like embedded systems, with memory constraints, this is not possible.
  \item Clone the context. This is the solution uses for the interface.\newline
\end{enumerate}
\begin{figure}[!ht]
\centering
%\frame{
% trim: left, bottom, right, up
\includegraphics[trim=0cm 18.5cm 12cm 0cm,
height=10.5cm]{figures/hash_signature_clone.pdf}
\caption{Context - clone example}
\label{fig:gci_clone}
%}
\end{figure}
As shown figure \ref{fig:gci_clone}, when we need to compute a result, but the
whole data added previously are needed for a future result, the solution is to
clone the context, meaning that the whole data added and the configuration is copied in another context.\newline
Then one context could be used to compute the result and the other one
to add other data when needed.\newline


\section{Key management}

\label{gci_key_mng}

The institute wants to use hardware-based cryptographic module with one of its
particularities of key management. The interface shall have a key management
too, so it can be possible to use it from the hardware-based cryptographic module.
The use of this key management is split into three parts. The possibility to put
a key coming externally, to get a key from the provider and to delete a key when
this one is not needed anymore. These functions will be implemented in the key
management in the interface too.


\subsection*{Put a key}
As said in chapter \ref{gci_gen_key} Generate key pair, when a key is generated,
the ID of the key is returned to the application.
This is because the key is saved in the interface, through the function `` Put
key ``.


\begin{figure}[!ht]
\centering
%\frame{
% trim: left, bottom, right, up
\includegraphics[trim=12cm 22cm 11.5cm 0cm]{figures/key_manag_put_key.pdf}
\caption{Key management - put a key}
\label{fig:gci_key_mng_put}
%}
\end{figure}

As shown on figure \ref{fig:gci_key_mng_put}, when a key should be stored in the
interface, the key is added to the function. The interface saves it in a
specific place and return the ID of where is the key stored.


\subsection*{Get a key}
When the key should be used to be sent to another peer or to compute
something like a signature, it should be possible to get it.

\begin{figure}[!ht]
\centering
%\frame{
% trim: left, bottom, right, up
\includegraphics[trim=12cm 22cm 11.5cm 0cm]{figures/key_manag_get_key.pdf}
\caption{Key management - get a key}
\label{fig:gci_key_mng_get}
%}
\end{figure}

As shown on figure \ref{fig:gci_key_mng_get}, when the key is needed by the
application, the ID should be sent the interface which will return the key
stored at this ID.

The key stays in the interface if it's still needed.

\subsection*{Delete a key}
When the key is not needed anymore, it should be possible to delete it to free
space for other keys.

\begin{figure}[!ht]
\centering
%\frame{
% trim: left, bottom, right, up
\includegraphics[trim=12cm 22cm 11.5cm 0cm]{figures/key_manag_del_key.pdf}
\caption{Key management - delete a key}
\label{fig:gci_key_mng_del}
%}
\end{figure}

As shown on figure \ref{fig:gci_key_mng_del}, when the key is not needed
anymore, the ID of it is sent to the interface, which will delete it.
