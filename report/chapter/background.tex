\chapter{Background}

Nowadays, the security of data is very important and widely use into several
domains, for example in Web browsers, e-mail, cell phones, bank cards, cars and
medical implants.
What is to understand under security is that the data remain
private (confidentiality), meaning that no one, except whom is intended the data
can understand the transmitted data, the data could not be modified
during the transmission (integrity), and the data is authenticated
(authentication), meaning that the sender can at any time be
authenticated by the receiver.
Several cryptographic algorithms are used today to obtain these specifications
and to have the most secured communication as possible.

The TLS protocol, which is today widely deployed as security protocol for all
kinds of Web-based applications for e-commerce and e-business for example, is
part of most security systems available today.

This protocol can use several different cryptographic algorithms to secure a
communication between a Client and a Server.
The different use of these algorithms are described into cipher suites, to use
the different function which can be provided by the TLS protocol.

Embedded systems, which are today more and more used because of this small size,
is a example of systems which TLS protocol must be optimized. The optimization
of this protocol is focus on the reduce of the calculation done for each
cryptographic algorithm, because most of the embedded systems are
powered by a battery, also with low autonomy, and embedded systems are slower
than ordinary computers.

