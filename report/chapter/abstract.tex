\chapter*{Abstract}

\section*{English}

This thesis presents the design and the implementation of a new Generic
Cryptographic Interface.

This interface is a generic software interface, an API, that allows
to transparently exchange cryptographic engines. Its purpose is to be
implemented on projects which need Cryptography, as the \embtls project,
developed in the Institute of Reliable Embedded Systems and Communications
Electronics (ivESK) in the Offenburg University of Applied Sciences.
The cryptographic engines, or providers, can be open-source cryptographic
software libraries (i.e \tomcrypt \cite{doc:tomcrypt}) or hardware-based
cryptographic modules (i.e \vaultic \cite{doc:vault}).

The \embtls project is based on secured communication between a Client and a
Server. This secured communication is done with the TLS protocol, widely uses
today. The particularity of this project is to develop a TLS protocol for
embedded systems, which are slower, have less autonomy and fewer memories as
an ordinary computer.

This thesis presents the design of this new interface and its implementation to
the \embtls project with the use of \tomcrypt library as a provider for the cryptographic calculations.

\section*{German}
Diese Arbeit stellt die Herstellung und Implementierung von einem generischen kryptographischen
Interface.

Dieses Interface ist ein API, benutzt, um in kryptographischen Projekte zu implementiert sein, wie
der \embtls Projekt, welche in der Institut f\"{u}r verl\"{a}ssliche Embedded
Systems und Kommukationselektronik (ivESK) in der Hochschule
Offenburg hergestellt ist.
Dieses Interface kann unterschiedlichen Provider benutzen, dessen f\"{u}r die Berechnung von die
kryptographischen Algorithmen benutzen ist. Der Provider kann open-source
kryptographische Software library (z.b \tomcrypt \cite{doc:tomcrypt}) oder
hardware-based kryptographische Module (z.b \vaultic \cite{doc:vault}) sein.

Der \embtls Projekt ist auf Sicherheit der Kommunikation zwischen ein Client und
ein Server basiert. Diese sicherte Kommunikation ist mit der TLS Protocol
gemacht, welche heute sehr benutzt ist. Die Besonderheit des \embtls Projekts
ist, um auf Embedded Systems zu funktionieren. Die Embedded Systems sind
langsamer, haben wenige Autonomie und geringere Speicher als die Computer, die man heute benutzt.

Diese Arbeit stellt die Herstellung des Interfaces, welche in der \embtls Projekt
implementiert ist, mit der Benutzung von der LibTomCrypt library f\"{u}r die
Berechnung der kryptographischen Algorithmen benutzen ist.
