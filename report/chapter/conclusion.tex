\chapter{Conclusion}
Through this project I get a lot of knowledge about cryptography. It allows me
to learn all the main cryptography algorithms use in cryptography, the
functioning of them and why these algorithms are used today.

I learned a lot about the TLS protocol too. It allows me to understand all the
step needed to get a secured communication, and that a lot of verification has
to be done to be sure that the communication is secure.


\section{Achieved work}

The project is, unfortunately, completely achieved, but globaly the most
important parts are done.

The project was split into four parts.
\begin{enumerate}[noitemsep]
  \item Get the knowledge about the main cryptography part and where it could be
  used in the TLS protocol
  \item Design of the new interface
  \item Implementation of the interface into the \embtls application
  \item Implementation of the \tomcrypt provider into the Generic Cryptographic
  application
\end{enumerate}

The achieved work in this project is the design of the new Generic
Cryptographic Interface (GCI)), and globally the implementation of the interface
in the application and the implementation of the provider in the interface.

This is not completely done, because all the cipher suites shown in the
chapter \ref{res} doesn't work in all case.
There is two parts in the project, when the application, \embtls, work as Client
and as Server.

The figure \ref{fig:res} shown the result for \embtls as Client. When \embtls
work as Server, all the cipher suites with the Diffie-Hellman and the Elliptic
Curve Diffie-Hellman (ECDH) don't work. The problem should come in the Server
Key Exchange part, because this is the only part which changed compared to the
other cipher suites.

\section{Future work}
The future work which could be done with this project would be to finish the
implementation of the \embtls as Server (the Diffie-Hellman and Elliptic Curve
Diffie-Hellman, which don't work), the implementation of the ECDSA algorithm as
key exchange, which was done in another project, the test of other cipher suites
to increase the use of this new Generic Cryptographic Interface (GCI), and the
most important would be to write the documentation of the new Generic Cryptographic Interface,
for other future projects which will need this interface as support.