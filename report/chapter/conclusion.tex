\chapter{Conclusion}


\section{Achieved work}

The project is, unfortunately, not completely achieved, but globally the most
important parts are done.

The project was split into four parts.
\begin{enumerate}[noitemsep]
  \item Get the knowledge about the main cryptography part and where it could be
  used in the TLS protocol
  \item Design of the new interface
  \item Implementation of the interface into the \embtls application
  \item Implementation of the \tomcrypt provider into the Generic Cryptographic
  application
\end{enumerate}

The achieved work in this project is the design of the new Generic
Cryptographic Interface (GCI)), and globally the implementation of the interface
in the application and the implementation of the provider in the interface.

These implementations were done in two parts. The first part when the \embtls
application works as Client. The Server was done with project OpenSSL
\cite{doc:openssl}.
Figure \ref{fig:res} shows the cipher suites when the application \embtls works as
Client.
The second part of the implementation was when the \embtls application works as
Server. The Client was done with the project cURL \cite{wiki:curl}. The cipher
suites which used Diffie-Hellman and Elliptic Curve Diffie-Hellman as key exchange don't work in
this part. The problem comes when we receive the Client Key Exchange. The secret
key, we have to compute, is not the same as this coming from the Client. That
should come from the decryption of the Client Key Exchange with the private key, which
the public key is sent in the Certificates. The problem could come when we send
the Server Key Exchange, maybe there is a shift with the sent data and the
Client receives the wrong public key\ldots

\section{Future work}

The future works which could be done with this project would be to finish the
implementation of the \embtls as Server (the Diffie-Hellman and Elliptic Curve
Diffie-Hellman protocols for the key exchange, which don't work), the
implementation of the ECDSA algorithm as key exchange, which was done in another
project, the test of other cipher suites to increase the use of this new Generic
Cryptographic Interface (GCI), the test of the interface in other projects to
use all the function provides by the interface and be sure that it doesn't miss
something and the most important would be to write the documentation of the new
Generic Cryptographic Interface, for other future projects which will need this
interface as support.

\section{What I learned}
Through this thesis I learned the Cryptography, a domain I didn't know until
this thesis, or maybe only the use of the hash for the checksum. I learned the
existing main cryptography algorithms and the use of them.

I learned about the TLS protocol too. It allows me to understand all the step
needed to get a secured communication and where is the TLS protocol used today.

I learned to use the software Wireshark, which was very useful for the
implementation part, chapter \ref{imp}.
